\documentclass[12pt, oneside]{book}
\usepackage{amsmath, amssymb, amssymb, mathtools, graphicx, hyperref, titling}
\usepackage{tikz, caption, subcaption, enumitem, polyglossia}
\usepackage{tikz-3dplot, float, etoolbox, enumitem}
\usepackage[top=25mm, bottom=25mm, left=20mm, right=20mm]{geometry}
%\usepackage{showframe}

\makeatletter
\patchcmd{\@makechapterhead}{\vspace*{50\p@}}{}{}{}% Removes space above \chapter head
\patchcmd{\@makeschapterhead}{\vspace*{50\p@}}{}{}{}% Removes space above \chapter* head
\DeclarePairedDelimiter{\bra}{\langle}{|}
\DeclarePairedDelimiter{\ket}{|}{\rangle}

\makeatother

\predate{}
\postdate{}
\date{}
\title{Quantum Mechanics 2}
\author{Nachiketa Kulkarni}
\pagenumbering{gobble}
\setmainfont{Comic Sans MS}

\begin{document}
\maketitle
\tableofcontents

\mainmatter
\chapter{Harmonic Oscilator}
Harmonic Oscilator is one of the few problems which is completely solvable in QM.
It is used to approximate a function at a minima.

The potential is as follows:
\[ V(x) = \frac{1}{2} m \omega^2 x^2 \]
In classical mechanics, the general solution for a harmonic oscilator is as follows:
\[x(t) = A \cos \omega t + \sin \omega t \]
We can differentiate accordingly and find the momentum of the particle.
The entire system is defined by the initial condition of the particle.

Now, coming to quantum mechanics, we will assume a wavefunction, \(\ket{\psi}\).
The corresponding Hamiltonian, \(H\) will be as follows:
\[H = \frac{p^2}{2m} + \frac{1}{2} m \omega^2 x^2\]


\end{document}
