\documentclass[12pt, oneside]{book}
\usepackage{amsmath, amssymb, amssymb, mathtools, graphicx, hyperref, titling}
\usepackage{tikz, caption, subcaption, enumitem, polyglossia}
\usepackage{tikz-3dplot, float, etoolbox, enumitem}
\usepackage[top=25mm, bottom=25mm, left=20mm, right=20mm]{geometry}
%\usepackage{showframe}

\makeatletter
\patchcmd{\@makechapterhead}{\vspace*{50\p@}}{}{}{}% Removes space above \chapter head
\patchcmd{\@makeschapterhead}{\vspace*{50\p@}}{}{}{}% Removes space above \chapter* head
\makeatother

\DeclarePairedDelimiter{\evdel}{\langle}{\rangle}
\newcommand{\ev}{\evdel}


\predate{}
\postdate{}
\date{}
\title{MT2213 - Group Theory}
\author{Nachiketa Kulkarni}
\pagenumbering{gobble}
\setmainfont{Comic Neue}

\begin{document}
\maketitle
\tableofcontents

\mainmatter
\chapter{Definitions}
\section{Groups}
A non-empty set G is a group, is considered to be a group with an operation \(\star\) if to every pair \( \left(x,y\right) \in G \times G\) and element \(x \star y \in G\) is assigned, satisfying the following axioms:
\begin{enumerate}
    \item \textbf{Associativity:} \(\forall\text{ } x,y,z \in G\), \(x \star \left(y \star z\right) = \left(x \star y \right) \star z = x \star y \star z \)
    \item \textbf{Existence of Identity:} There exists an element \(e \in G\) such that \(e \star g = g \star e = g\)
    \item \textbf{Existence of Inverse:} For every element \(x \in G\) there exists an element \(x^{-1} \in G\) such that \(x \star x^{-1} = e = x^{-1} \star x \), where \(e \in G\) is the identity element of the group.
\end{enumerate}
It is represented as \( \left(G, \star \right) \).
Some properties of groups:
\begin{enumerate}
    \item \textbf{Uniqueness of Identity:} The identity element of a group is unique.
    Consider \(e_1, e_2 \in G, e_1 \neq e_2\) and both are identity elements.
    Let \(x \in G\), then \( e_1 \star x = e_2 \star x = x \).
    This also implies that \(e_1 = e_2\), hence the identity element is unique.
    \item \textbf{Uniqueness of Inverse:} The inverse of an element in a group is unique.
    Consider \(x \in G\), and \(y_1, y_2 \in G\) are inverses of \(x\).
    Then, \(x \star y_1 = e = y_1 \star x\) and \(x \star y_2 = e = y_2 \star x\).
    Now, \(y_1 = y_1 \star e = y_1 \star \left(x \star y_2\right) = \left(y_1 \star x\right) \star y_2 = e \star y_2 = y_2\).
    Hence, the inverse of an element is unique.
\end{enumerate}

\subsection{Examples:}
\begin{enumerate}
    \item \( \left( \mathbb{Z}, +\right) \) is a group:
    \begin{enumerate}
        \item Associativity: Addition is associative.
        \item Identity: \(0\) is the identity. Let \(x \in Z\). Now \(0 + x = x + 0 = x\). Hence, it is an identity.
        \item Inverse: Let \(x \in \mathbb{Z}\). Now, \(x + (-x) = (-x) + x = 0\), where \(0\) is the additive identity.
    \end{enumerate}
    \item \(\left( \mathbb{Q}^{+}, \times \right)\) is a group:
    \begin{enumerate}
        \item Associativity: Multiplication is associative.
        \item Identity: \(1\) is the identity:
        Let \(x \in \mathbb{Q}^{+} \). Now, \(1 \times x = x \times 1 = x\). Hence, it is an identity.
        \item Inverse: Let \(x \in \mathbb{Q}^{+} \), Now, \(x \times \frac{1}{x} = \frac{1}{x} \times x = 1\), where 1 is the multiplicative identity.
    \end{enumerate} \break
    \item \(\left(GL(n,\mathbb{R}), \times \right)\) is a group, where \(\times\) is matrix multiplication (or combination of linear transformations):
    \begin{enumerate}
        \item Associativity: Matrix multiplication is associative.
        \item Identity: \(I_n\) is the identity matrix.
        \item Inverse: Let \(A \in GL(n,\mathbb{R})\), then \(A \times A^{-1} = A^{-1} \times A = I_n\).
    \end{enumerate}
\end{enumerate}
Check if:
\begin{enumerate}
    \item \( \left(\mathbb{R}, \times \right)\) is a group or not.
    
    \(0 \in \mathbb{R}\), \(0\) does not have an inverse. Hence, it is not a group.

    \item \(\left(\mathbb{C}, \times\right)\) is a group or not.
    
    \(0 \in \mathbb{C}\), \(0\) does not have an inverse. Hence, it is not a group.
    
    \item \(\left(\mathbb{R}/\{0\}, \times \right)\) is a group or not.
    
    Yes its a group:
    \begin{enumerate}
        \item Associativity: Multiplication is associative.
        \item Identity: 1 is an identity:
        Let \(x \in \mathbb{R}/\{0\} \). Now, \(1 \times x = x \times 1 = x\). Hence, it is an identity.
        \item Inverse: Let \(x \in \mathbb{R}/\{0\} \), Now, \(x \times \frac{1}{x} = \frac{1}{x} \times x = 1\), where 1 is the multiplicative inverse.
    \end{enumerate}

    \item \(\left(\mathbb{C}/\{0\}, \times \right)\) is a group or not.
    
    Yes it is a group:
    \begin{enumerate}
        \item Associativity: Multiplication is associative.
        \item Identity: 1 is an identity:
        Let \(x \in \mathbb{C}/\{0\} \). Now, \(1 \times x = x \times 1 = x\). Hence, it is an identity.
        \item Inverse: Let \(x \in \mathbb{C}/\{0\} \), Now, \(x \times \frac{1}{x} = \frac{1}{x} \times x = 1\), where 1 is the multiplicative inverse.
    \end{enumerate}
\end{enumerate}
\subsection{Abelian Groups}
A group \( \left(G, \star \right) \) is said to be abelian if the operation \(\star\) is commutative, i.e., \(x \star y = y \star x\), \(\forall x,y \in G\).

\subsection{Conjugate}
Consider a group \( \left(G, \star \right) \).
For \(x,y \in G\), \(y\) is said to be conjugate of \(x\) if there exists an element \(a \in G\) such that:
\[ y = a \star x \star a^{-1} \]
\paragraph{Note:} For a given \(a\), the conjugate of \(x\) is unique.
i.e., if we consider conjugate to be a function \(f_a\), then \(f_a\) is a bijection.

\subsection{Order of a Group}
The order of a group \(G\) is the number of elements in the group.
It is denoted by \(|G|\).
A group \(G\) is said to be finite if the number of elements in it is finite.
Otherwise, it is said to be infinite.

\subsection{Cyclic Group}
A group \( \left(G, \star \right) \) is said to be cyclic if there exists an element \(a \in G\) such that every element of \(G\) can be written as a power of \(a\).
Let, \(G = \ev{g}\) by a cyclic group of order \(n\).
Then, \(G = \{e, g, g^2, \ldots, g^{n-1}\}\).
\subsubsection{Properties of Cyclic Groups}
\begin{enumerate}
    \item All cyclic groups are abelian.
    \item \(n = \min{ \left\{m \in \mathbb{N} \: \vline \: g^m = 1\right\} }\).

    \paragraph{Proof:} As the order of \(G\) is finite, there exists \(a,b \in \mathbb{N}\) such that \(g^a = g^b\). 
    This implies: \(g^{a-b} = 1\).
    \[ \therefore \exists \: n := \min{ \left\{m \in \mathbb{N} \: \vline \: g^m = 1\right\} } \]

    \item If \(z \in \mathbb{Z}\): \(g^z = 1 \implies n \: \vline \: z \).
    \paragraph{Proof:} Let \(z = qn + r\), where \(0 \leq r < n\).
    \[ g^z = g^{qn+r} = g^{qn} \star g^r = 1 \star g^r = g^r \]
    If \(r\) is such that \(g^r = 1\), then \(r = 0\).
    Hence, \(n \: \vline \: z\).
    \item For \(i,j,k \in \left\{1,2,3,\dots,n-1\right\}\): \(g^i \star g^j = g^k \implies i+j \equiv k \pmod{n}\).
\end{enumerate}

\subsection{Sub-groups}
Consider a group \( \left(G, \star \right) \).
A non-empty subset \(H\) is a subgroup of \(G\) if \(H\) is a group with the same operation \(\star\) as \(G\).
It is represented as \(H \leq G\).

A few properties of subgroups:
\begin{enumerate}
    \item \textbf{Identity:} The identity element of \(G\) is also the identity element of \(H\).
    \item \textbf{Inverse:} If \(x \in H\), then \(x^{-1} \in H\).
\end{enumerate}
Every group has at least two subgroups: the trivial subgroups \(U = \{e\}\) and the group itself \(G\).
We abuse notation and simply write \(U = 1\).

\end{document}
