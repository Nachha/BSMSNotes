\documentclass[12pt, oneside]{book}
\usepackage{amsmath, amssymb, amssymb, mathtools, graphicx, hyperref, titling}
\usepackage{tikz, caption, subcaption, enumitem, polyglossia}
\usepackage{tikz-3dplot, float, etoolbox, enumitem}
\usepackage[top=25mm, bottom=25mm, left=20mm, right=20mm]{geometry}
%\usepackage{showframe}

\makeatletter
\patchcmd{\@makechapterhead}{\vspace*{50\p@}}{}{}{}% Removes space above \chapter head
\patchcmd{\@makeschapterhead}{\vspace*{50\p@}}{}{}{}% Removes space above \chapter* head
\makeatother

\predate{}
\postdate{}
\date{}
\title{BI2233 - Genetics}
\author{Nachiketa Kulkarni}
\pagenumbering{gobble}
\setmainfont{Comic Neue}

\begin{document}
\maketitle
\tableofcontents

\mainmatter
\chapter{Introduction}
Some of the principles of Genetics has been known for a very long time.
Eg: If a specific feature was required from a plant/animal, one would bread only those with the features and hence obtain individuals that have the same traits
Genetics as a formal topic is only about 200 years old.

\paragraph{Tangent 1:} There are many biological phenomena that were understood using genetics.
\textit{Drosophila} was used to understand Cancer, aging, etc.
The Central Dogma of Molecular Biology was mainly understood using single-celled organisms (\textit{E. Coli}) 

\paragraph{Tangent 2:}Aging is a phenomena that is very well known.
How to identify if a gene (or a set of genes) influence aging?
One could start by trying to mutate certain genes and see if the lifespan of the individual changes.
(Japanese have a relatively longer lifespan).
The age of a person may also be affected by various environmental factors.
There could also be many epigenetic factors that are not directly related to the genetic information.

\section{Mendelian Genetics}
Mendel used the garden pea plant to understand the principles of genetics.
He looked at 7 different traits of the plant.

\subsection{Experimental Setup}
Mendel got lucky with the his choice of plant.
He began by making his control individuals by breeding.
The method for growing pea plants is as follows:
\begin{enumerate}
    \item Seeds are planted in spring.
    \item Flowers appear in the summer.
    \item Fertilization is carried out in the summer, either self-fertilization or cross-fertilization.
    \item %fill in details later
\end{enumerate}
This shows that, every year, one can get 1-2 generations of pea plants.

\subsection{Mono-hybrid Cross}
Mendel began by looking at one trait at a time.
Consider the trait of seed shape.
Initially, he took a plant that was homozygous for round seeds and another that was homozygous for wrinkled seeds.
This is the P1 generation.
He then crossed these two plants to get the F1 generation.
The F1 generation was all round seeds.
He then self-fertilized the F1 generation to get the F2 generation.
In the F2 generation, he observed that the ratio of round to wrinkled seeds was \(3 : 1\).
%add diagram here

\subsection{Di-hybrid Cross}
Mendel then looked at two traits at a time.

\subsection{Mendel's Laws}
From the above experiments, Mendel formulated the Laws of Genetics.
A few terms that were used by Mendel to explain his observations:
\begin{itemize}
    \item \textbf{Allele:} Different forms of a gene.
    \item \textbf{Dominant:} The allele that is expressed in the phenotype.
    \item \textbf{Recessive:} The allele that is not expressed in the phenotype.
    \item \textbf{Homozygous:} An individual that has two copies of the same allele.
    \item \textbf{Heterozygous:} An individual that has two different alleles.
\end{itemize}


\end{document}