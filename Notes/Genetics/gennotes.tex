\documentclass[12pt, oneside]{book}
\usepackage{amsmath, amssymb, amssymb, mathtools, graphicx, hyperref, titling}
\usepackage{tikz, caption, subcaption, enumitem, polyglossia}
\usepackage{tikz-3dplot, float, etoolbox, enumitem}
\usepackage[top=25mm, bottom=25mm, left=20mm, right=20mm]{geometry}
%\usepackage{showframe}

\makeatletter
\patchcmd{\@makechapterhead}{\vspace*{50\p@}}{}{}{}% Removes space above \chapter head
\patchcmd{\@makeschapterhead}{\vspace*{50\p@}}{}{}{}% Removes space above \chapter* head
\makeatother

\predate{}
\postdate{}
\date{}
\title{BI2233 - Genetics}
\author{Nachiketa Kulkarni}
\pagenumbering{gobble}
\setmainfont{Comic Neue}

\begin{document}
\maketitle
\tableofcontents

\mainmatter
\chapter{Introduction}
Some of the principles of Genetics has been known for a very long time.
Eg: If a specific feature was required from a plant/animal, one would bread only those with the features and hence obtain individuals that have the same traits
Genetics as a formal topic is only about 200 years old.

\paragraph{Tangent 1:} There are many biological phenomena that were understood using genetics.
\textit{Drosophila} was used to understand Cancer, aging, etc.
The Central Dogma of Molecular Biology was mainly understood using single-celled organisms (\textit{E. Coli}) 

\paragraph{Tangent 2:}Aging is a phenomena that is very well known.
How to identify if a gene (or a set of genes) influence aging?
One could start by trying to mutate certain genes and see if the lifespan of the individual changes.
(Japanese have a relatively longer lifespan).
The age of a person may also be affected by various environmental factors.
There could also be many epigenetic factors that are not directly related to the genetic information.


\end{document}