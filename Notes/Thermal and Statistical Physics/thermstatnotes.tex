\documentclass[12pt, oneside]{book}
\usepackage{amsmath, amssymb, amssymb, mathtools, graphicx, hyperref, titling}
\usepackage{tikz, caption, subcaption, enumitem, polyglossia}
\usepackage{tikz-3dplot, float, etoolbox, enumitem}
\usepackage[top=25mm, bottom=25mm, left=20mm, right=20mm]{geometry}
%\usepackage{showframe}

\usepackage{mathtools}
\DeclarePairedDelimiter{\evdel}{\langle}{\rangle}
\newcommand{\ev}{\evdel}

\makeatletter
\patchcmd{\@makechapterhead}{\vspace*{50\p@}}{}{}{}% Removes space above \chapter head
\patchcmd{\@makeschapterhead}{\vspace*{50\p@}}{}{}{}% Removes space above \chapter* head
\makeatother

\predate{}
\postdate{}
\date{}
\title{PH2223 - Thermal and\\Statistical Physics}
\author{Nachiketa Kulkarni}
\pagenumbering{gobble}
\setmainfont{Comic Neue}

\begin{document}
\maketitle
\tableofcontents

\mainmatter
\chapter{Preliminaries}
\section{Introduction}
\subsection{Large Numbers - What is a mole?}
Population of India: \( \approx 1.4 \times 10^9\)\\
Indian Economy: \( \approx 4 \times 10^{12} \) USD\\
Number of \(\text{N}_2\) molecules in 1kg of Nitrogen: \( \approx 10^{25} \)\\

This number is too big.
Instead of dealing with numbers this large.
We will be dealing with averages for these large set of numbers.

Consider the addition of two numbers \(a_1\) and \(a_2\), given they are of the same order of magnitude.
As we increase the number of terms, the sum increases, but overall sum is not going to change by a significant value.
As we approach \(100\) terms, another term added will only result in minor fluctuations in the sum. 

We use the term mole as a unit to represent the number of particles in a system.
\paragraph{Mole:}A mole is defined as the quantity of matter that contains as many entities as the number of atoms in exactly \(12\text{g}\) of \(^{12}\text{C}\).
This number is represented as \(N_A\) and is approximated to:
\[ N_A = 6.022 \times 10^{23} \]

\paragraph{Thermodynamic Limit:} When the number of elements in our system increases to a number large enough such that the fluctuations in the sum is very little is known as the Thermodynamic Limit.
This is the point where we can deal with averages and their distributions instead of precise values.

\subsection{Properties of Gas}
There are many properties that can be used to describe the state of a Gas.
Eg: Volume \(V\), Internal Energy \(U\), Pressure \(P\), Temperature \(T\).

These properties are broadly classified as:
\begin{enumerate}
    \item \textbf{Intensive Properties} Properties that are not affected by the size of the system. Eg: \(P, T\)
    \item \textbf{Extensive Properties} Properties that scale with the size of system. Eg: \(V, U\)
\end{enumerate}

\subsection{Ideal Gas}
Experiments on gases show some of the relations between \(P\), \(V\) and \(T\)
\begin{enumerate}
    \item \textbf{Boyle's Law} A fixed amount of gas obeys:
    \[P \propto \frac{1}{V}\]
\end{enumerate}

\section{Heat}
\subsection{Definition}
Heat is defined as the thermal energy in transit.
Heat is never stored in objects.
It is only energy that is stored.
Heat naturally only flows from a body with higher temperature to a body with lower temperature.
Though, when there is an external energy is provided, heat can flow from a colder body to a hotter body.
\subsection{Heat Capacity}
Heat is not stored.
But what Physics has conventionally called this Heat Capacity, so we will call it Heat Capacity.
\paragraph{Heat Capacity} It is the amount of heat(energy) \(dQ\) that is required to raise the temperature of the system by a small amount \(dT\).
\[\text{Heat Capacity }C = \frac{dQ}{dT} \]
Heat Capacity of the system \(C\) per unit mass is known as the Specific Heat Capacity \(c\).

Often times the heat provided to the system isn't utilized fully just to increase temperature.
There are often other constraints to the system, like if the heat is used to do work on the surroundings.
In these cases, we will be using terms like Heat Capacity at constant pressure \(C_P\) or Heat Capacity at constant Volume \(C_V\)

\section{Probability}
Probability is required in Statistical Mechanics because a lot of the phenomenon that occur predicted by the physics are often with a given with a statistical background.
Eg: Probability of various macrostates of a system.

\subsection{Discrete Probability Distributions}
Discrete Random Variables take a finite (or countably infinite) values.
Some of the properties include:
\begin{enumerate}
    \item Sum of probabilities of all random variables add up to \(1\):
    \[\sum_{i} p_i = 1\]
    \item Mean or expectation value of the random variable is defined as:
    \[ \ev{x} = \sum_{i} x_i p_i \]
\end{enumerate}
\end{document}