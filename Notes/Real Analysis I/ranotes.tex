\documentclass[12pt, oneside]{book}
\usepackage{amsmath, amssymb, amssymb, mathtools, graphicx, hyperref, titling}
\usepackage{tikz, caption, subcaption, enumitem, polyglossia}
\usepackage{tikz-3dplot, float, etoolbox, enumitem}
\usepackage[top=25mm, bottom=25mm, left=20mm, right=20mm]{geometry}
%\usepackage{showframe}

\makeatletter
\patchcmd{\@makechapterhead}{\vspace*{50\p@}}{}{}{}% Removes space above \chapter head
\patchcmd{\@makeschapterhead}{\vspace*{50\p@}}{}{}{}% Removes space above \chapter* head
\makeatother

\predate{}
\postdate{}
\date{}
\title{MT2223 - Real Analysis I}
\author{Nachiketa Kulkarni}
\pagenumbering{gobble}
\setmainfont{Comic Neue}

\begin{document}
\maketitle
\tableofcontents

\mainmatter
\chapter{Real Number System}
\section{Natural Numbers}
\( \mathbb{N} = \{ 1,2,3,\dots \} \) is the set of all natural numbers.
They can also be referred to as Positive Integers.
The set of Natural Numbers is important because it is the smallest set which is closed under addition and multiplication.

\subsection{Peano Axioms}
\begin{enumerate}
    \item The number \(1 \in \mathbb{N}\)
    \item For every Natural Number \(n\), there exists another Natural Number \(m\) which is known as the successor of \(n\).
    \item \(1\) is not the successor for any number in the set of Natural Number set.
    \item If \(m,n \in \mathbb{N}\) have the same successor, then \(m=n\).
    \item If \(A \subset \mathbb{N}\) such that \(1 \in A\) and \(n \in A \implies n+1 \in A\), then \(A = \mathbb{N}\).
\end{enumerate}

\section{Integers}
\( \mathbb{Z} = \{ \dots, -2, -1, 0, 1, 2, \dots \} \) is the set of all integers.
This set is important because it is the smallest set which is closed under subtraction.

\section{Rational Numbers}
The set of Rational Numbers is defined as:
\[ \mathbb{Q} = \left\{ \frac{m}{n} \; \vline \; m,n \in \mathbb{Z}, n \neq 0 \right\} \]
This set is important because it is the smallest set which is closed under division.

\section{Real Numbers}
We can see that the set of Rational Numbers is incomplete.
There is no Rational Number whose square is \(2\).
\paragraph{Proof:} Consider a rational number whose square is \(2\).
\[ \left(\frac{p}{q}\right)^2 = 2 \]
Here, \(p\) and \(q\) are both integers and co-prime.
\begin{align*}
    \frac{p^2}{q^2} &= 2 \\
    p^2 &= 2q^2
\end{align*}
Now, as \(p^2\) is even, \(p\) must be even.
Let \(p = 2k\). then, \(p^2 = 4k^2\): 
\begin{align*}
    4k^2 &= 2q^2 \\
    2k^2 &= q^2
\end{align*}
Now, as \(q^2\) is even, \(q\) must be even.

But this contradicts the fact that \(p\) and \(q\) are co-prime.
Hence, by proof of contradiction, there is no Rational Number whose square is \(2\).

\paragraph{Question:} \(A = \left\{ p \in \mathbb{Q} \; | \; p^2 < 2 \right\}\).
Can we find a number \(q \in A\) such that \( p \leq q, \forall p \in A \)? 
\paragraph{Answer:} Claim: given \(p \in A\), we can find \(q \in A\) such that \(p < q\).
Consider the following number:
\begin{align*}
    q &= p + \frac{2-p^2}{2+p} \\
    q &= \frac{p\left(2+p\right) + 2 - p^2}{p +2}\\
    q &= \frac{2 + 2p}{2 + p} \\
    q^2 - 2 &= \frac{2\left(p^2 - 2\right)}{\left(p + 2\right)^2}
\end{align*}
Looking at the first equation \(q > p\). If \(p^2 < 2\), then the final equation shows that \(q^2 < 2\).

\section{Functions}
A function from a set \(A\) to a set \(B\) is a subset of \(A \times B\) such that for every element in \(A\), there exists a unique element in \(B\).
The set \(A \times B \) is known as the Cartesian Product of \(A\) and \(B\) or Relation from \(A\) to \(B\).
A function from \(A\) to \(B\) is denoted as \(f: A \to B\).

\section{Ordered Sets}
Let \(S\) be a ordered set.
An order on \(S\) is a relation, denoted by \(<\), such that:
\begin{enumerate}
    \item For every \(x,y \in S\), exactly one of the following is true:
    \begin{itemize}
        \item \(x < y\)
        \item \(x = y\)
        \item \(y > x\)
    \end{itemize}
    \item \(\forall \: x,y,z \in S\) if \(x < y\) and \(y < z\), then \(x < z\).
\end{enumerate}

\subsection{Upper and Lower Bounds}
Let \(S\) be an ordered set and \(E \subset S\).
\begin{enumerate}
    \item \(E\) is said to be bounded above if there exists \( a \in S \), such that for every \( x \in E \), \( x \leq a \). Here, \(a\) is called an Upper Bound for \(E\).
    \item \(E\) is said to be bounded below if there exists \( b \in S \), such that for every \( x \in E \), \( x \geq b \). Here, \(b\) is called a Lower Bound for \(E\).
\end{enumerate}

\subsubsection{Supremum and Infimum}
\paragraph{Supremum:} Assume \(E\) to be bounded above.
Suppose there exists a number \(\alpha \in S\) such that:
\begin{itemize}
    \item \(\alpha\) is an Upper Bound for \(E\).
    \item If \(\gamma < \alpha\), then \(\gamma\) is not an Upper Bound for \(E\).
\end{itemize}
Then \( \alpha \) is known as the least Upper Bound or Supremum of \(E\).
\paragraph{Infimum:} Assume \(E\) to be bounded below.
Suppose there exists a number \(\beta \in S\) such that:
\begin{itemize}
    \item \(\beta\) is a Lower Bound for \(E\).
    \item If \(\gamma > \beta\), then \(\gamma\) is not a Lower Bound for \(E\).
\end{itemize}
Then \( \beta \) is known as the greatest Lower Bound or Infimum of \(E\).

\subsection{Least Upper Bound Property}
An ordered set \(S\) is said to have the Least Upper Bound Property if \(E \subset S\) is non-empty and bounded above, then \( \sup E\) exists in \(S\).

\subsubsection{Theorem}
Let \(S\) be an ordered set with the Least Upper Bound Property.
Let \(B \subset S\) be non-empty and bounded below.
Let \(L\) be the set of all Lower Bounds of \(B\).
Then, \( \alpha = \sup L \) exists in \(S\), and \(\alpha = \inf B\).

\paragraph{Proof:} As \(B\) is bounded below, \(L\) is non-empty.
The set \(L\) is bounded above by any element of \(B\).

Let \( \alpha \in S \: \& \: \alpha = \sup L \).
Now, if \( \gamma < \alpha \), then \(\gamma\) is not an Upper Bound for \(L\), and hence \( \gamma \notin B \).
It is also true that \( \alpha \leq x \) for every \( x \in B \).
Hence, \( \alpha \in L \).

If \( \alpha < \beta \) then \( \beta \notin L \), since \( \alpha \) is an upper bound of \(L\).
We have shown that \( \alpha \in \) but \(\beta \notin L\) if \( \alpha < \beta \).
Therefore, \( \alpha \) is a lower bound for \(B\) and \(\beta\) is not.
This implies that \( \alpha = \inf B \).
\end{document}