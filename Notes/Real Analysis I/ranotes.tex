\documentclass[12pt, oneside]{book}
\usepackage{amsmath, amssymb, amssymb, mathtools, graphicx, hyperref, titling}
\usepackage{tikz, caption, subcaption, enumitem, polyglossia}
\usepackage{tikz-3dplot, float, etoolbox, enumitem}
\usepackage[top=25mm, bottom=25mm, left=20mm, right=20mm]{geometry}
%\usepackage{showframe}

\makeatletter
\patchcmd{\@makechapterhead}{\vspace*{50\p@}}{}{}{}% Removes space above \chapter head
\patchcmd{\@makeschapterhead}{\vspace*{50\p@}}{}{}{}% Removes space above \chapter* head
\makeatother

\predate{}
\postdate{}
\date{}
\title{MT2223 - Real Analysis I}
\author{Nachiketa Kulkarni}
\pagenumbering{gobble}
\setmainfont{Comic Neue}

\begin{document}
\maketitle
\tableofcontents

\mainmatter
\chapter{Real Number System}
\section{Natural Numbers}
\( \mathbb{N} = \{ 1,2,3,\dots \} \) is the set of all natural numbers.
They can also be referred to as Positive Integers.
The set of Natural Numbers is important because it is the smallest set which is closed under addition and multiplication.

\subsection{Peano Axioms}
\begin{enumerate}
    \item The number \(1 \in \mathbb{N}\)
    \item For every Natural Number \(n\), there exists another Natural Number \(m\) which is known as the successor of \(n\).
    \item \(1\) is not the successor for any number in the set of Natural Number set.
    \item If \(m,n \in \mathbb{N}\) have the same successor, then \(m=n\).
    \item If \(A \subset \mathbb{N}\) such that \(1 \in A\) and \(n \in A \implies n+1 \in A\), then \(A = \mathbb{N}\).
\end{enumerate}

\section{Integers}
\( \mathbb{Z} = \{ \dots, -2, -1, 0, 1, 2, \dots \} \) is the set of all integers.
This set is important because it is the smallest set which is closed under subtraction.

\section{Rational Numbers}
The set of Rational Numbers is defined as:
\[ \mathbb{Q} = \left\{ \frac{m}{n} \; \vline \; m,n \in \mathbb{Z}, n \neq 0 \right\} \]
This set is important because it is the smallest set which is closed under division.

\section{Real Numbers}
We can see that the set of Rational Numbers is incomplete.
There is no Rational Number whose square is \(2\).
\paragraph{Proof:} Consider a rational number whose square is \(2\).
\[ \left(\frac{p}{q}\right)^2 = 2 \]
Here, \(p\) and \(q\) are both integers and co-prime.
\begin{align*}
    \frac{p^2}{q^2} &= 2 \\
    p^2 &= 2q^2
\end{align*}
Now, as \(p^2\) is even, \(p\) must be even.
Let \(p = 2k\). then, \(p^2 = 4k^2\): 
\begin{align*}
    4k^2 &= 2q^2 \\
    2k^2 &= q^2
\end{align*}
Now, as \(q^2\) is even, \(q\) must be even.

But this contradicts the fact that \(p\) and \(q\) are co-prime.
Hence, by proof of contradiction, there is no Rational Number whose square is \(2\).
%Prove why there is no Rational Number whose square is 2.

\paragraph{Question:} \(A = \left\{ p \in \mathbb{Q} \; | \; p^2 < 2 \right\}\).
Can we find a number \(q \in A\) such that \( p \leq q, \forall p \in A \)? 
\paragraph{Answer:} Claim: given \(p \in A\), we can find \(q \in A\) such that \(p < q\).
Consider the following number:
\begin{align*}
    q &= p + \frac{2-p^2}{2+p} \\
    q &= \frac{p\left(2+p\right) + 2 - p^2}{p +2}\\
    q &= \frac{2 + 2p}{2 + p} \\
\end{align*}


\end{document}
