\documentclass[12pt, oneside]{book}
\usepackage{amsmath, amssymb, amssymb, mathtools, graphicx, hyperref, titling}
\usepackage{tikz, caption, subcaption, enumitem, polyglossia}
\usepackage{tikz-3dplot, float, etoolbox}
\usepackage[top=20mm, bottom=20mm, left=15mm, right=15mm]{geometry}
%\usepackage{showframe}

\makeatletter
\patchcmd{\@makechapterhead}{\vspace*{50\p@}}{}{}{}% Removes space above \chapter head
\patchcmd{\@makeschapterhead}{\vspace*{50\p@}}{}{}{}% Removes space above \chapter* head
\makeatother

\predate{}
\postdate{}
\date{}
\title{MT2123 - Advanced Linear Algebra}
\author{Nachiketa Kulkarni}
\pagenumbering{gobble}
\setmainfont{Times New Roman}

\begin{document}
\maketitle
\tableofcontents

\mainmatter
\chapter{Fields and Vector Spaces}
\section{Groups}
\paragraph{Definition} A group \( \langle G, \ast \rangle \) is a set \( G \) with a binary operation \( \ast \) such that the following axioms are satisfied:
\begin{enumerate}
    \item Closure: For all \( a, b \in G \), \( a \ast b \in G \).
    \item Associativity: For all \( a, b, c \in G \), \( a \ast (b \ast c) = (a \ast b) \ast c \).
    \item Identity Element: There exists an element \( I \in G \) such that for all \( I \in G \), \( a \ast I = I \ast a = a \). Here, \( I \) is called as the identity element of \( \ast \) in \( G \).
    \item Inverse: corresponding to every element \( a \in G \), there exists an element \( a' \in G \) such that \( a \ast a' = a' \ast a = I \). Here, \( a' \) is called as the inverse of \( a \) in \( G \).
\end{enumerate}
\section{Rings}
\paragraph{Definition} A ring \( \langle R, +, \cdot \rangle \) is a set \( R \) with two binary operations \( + \) and \( \cdot \), which we call addition and multiplication, such that the following axioms are satisfied:
\begin{enumerate}
    \item \( \langle R, + \rangle \) is an abelian/commutative group. 
    \item Multiplication is associative: For all \( a, b, c \in R \), \( a \cdot (b \cdot c) = (a \cdot b) \cdot c \).
    \item Distributive Property: For all \( a, b, c \in R \), the Left Distributive Law, \( a \cdot (b + c) = a \cdot b + a \cdot c \) and Right Distributive Law, \( (a + b) \cdot c = a \cdot c + b \cdot c \).
\end{enumerate}
\section{Fields}
\paragraph{Definition} A field \( \langle F, +, \cdot \rangle \) is a set \( F \) with two binary operations \( + \) and \( \cdot \), which we call addition and multiplication, such that the following axioms are satisfied:
\begin{enumerate}
    \item Closure: For all \( a, b \in F \), \( a + b \in F \) and \( a \cdot b \in F \).
    \item Associativity: For all \( a, b, c \in F \), \( a + (b + c) = (a + b) + c \) and \( a \cdot (b \cdot c) = (a \cdot b) \cdot c \).
    \item Commutativity: For all \( a, b \in F \), \( a + b = b + a \) and \( a \cdot b = b \cdot a \).
    \item Identity Elements: There exist two elements \(I, O \in F\) such that for all \( a \in F \), \(I \cdot a = a \) and \( O + a = a \). Here, \( I \) is called as the multiplicative identity and \( O \) is called as the additive identity.
    \item Additive Inverse: For all \( a \in F \), there exists an element \( -a \in F \) such that \( a + (-a) = O \). Here, \( -a \) is called as the additive inverse of \( a \).
    \item Multiplicative Inverse: For all \( a \neq O \in F \), there exists an element \( a^{-1} \in F \) such that \( a \cdot a^{-1} = I \). Here, \( a^{-1} \) is called as the multiplicative inverse of \( a \).
    \item Distributivity: For all \( a, b, c \in F \), the Left Distributive Law, \( a \cdot (b + c) = a \cdot b + a \cdot c \) and Right Distributive Law, \( (a + b) \cdot c = a \cdot c + b \cdot c \).
\end{enumerate}
\subsection{Detour 1 - Finite Fields}
another long wall of text
\chapter{Linear Transformations}
long wall of text incoming

\end{document}