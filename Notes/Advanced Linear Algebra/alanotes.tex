\documentclass[12pt, oneside]{book}
\usepackage{amsmath, amssymb, amssymb, mathtools, graphicx, hyperref, titling}
\usepackage{tikz, caption, subcaption, enumitem, polyglossia}
\usepackage{tikz-3dplot, float}
\usepackage[top=5mm, bottom=5mm, left=15mm, right=15mm]{geometry}
\predate{}
\postdate{}
\date{}
\title{MT2123 - Advanced Linear Algebra}
\author{Nachiketa Kulkarni}
\pagenumbering{gobble}

\begin{document}
\maketitle
\tableofcontents

\mainmatter
\chapter{Fields and Vector Spaces}
\section{Groups}
\paragraph{Definition} A group \( \langle G, \ast \rangle \) is a set \( G \) with a binary operation \( \ast \) such that the following axioms are satisfied:
\begin{enumerate}
    \item Closure: For all \( a, b \in G \), \( a \ast b \in G \).
    \item Associativity: For all \( a, b, c \in G \), \( a \ast (b \ast c) = (a \ast b) \ast c \).
    \item Identitiy Element: There exists an element \( I \in G \) such that for all \( I \in G \), \( a \ast I = I \ast a = a \). Here, \( I \) is called as the identity element of \( \ast \) in \( G \).
    \item Inverse: corresponding to every element \( a \in G \), there exists an element \( a' \in G \) such that \( a \ast a' = a' \ast a = I \). Here, \( a' \) is called as the inverse of \( a \) in \( G \).
\end{enumerate}
\section{Rings}
\paragraph{Definition} A ring \( \langle R, +, \cdot \rangle \) is a set \( R \) with two binary operations \( + \) and \( \cdot \), which we call addition and multiplication, such that the following axioms are satisfied:
\begin{enumerate}
    \item \( \langle R, + \rangle \) is an abelian/commutative group. 
    \item Multiplication is associative: For all \( a, b, c \in R \), \( a \cdot (b \cdot c) = (a \cdot b) \cdot c \).
    \item Distributive Property: For all \( a, b, c \in R \), the Left Distributive Law, \( a \cdot (b + c) = a \cdot b + a \cdot c \) and Right Distributive Law, \( (a + b) \cdot c = a \cdot c + b \cdot c \).
\end{enumerate}
\section{Fields}
\paragraph{Definition}  
\chapter{Linear Transformations}
long wall of text incoming

\end{document}