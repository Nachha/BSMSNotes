\documentclass{article}
\usepackage{amsmath, amssymb, amssymb}

\begin{document}
\begin{center}
    \begin{LARGE}
        \textbf{Linear Algebra}
    \end{LARGE}
\end{center}

\section{Introduction to Linear Systems}
\index{Introduction to Linear Systems}
They are system of equations that have variables that are linear.
Example: $$x + y = 2$$ and $$2x - y = 1$$

Normally the coefficients are real numbers, but they can be complex numbers as well. \\[1pt]
System of linear equations of $m$ equations and $n$ variables:
    $$
    a_{1,1} x_1 + a_{1,2} x_2 + \cdots + a_{1,n} x_n = b_1
    $$
    $$
    a_{2,1} x_1 + a_{2,2} x_2 + \cdots + a_{2,n} x_n = b_2
    $$
    $$
    \vdots
    $$
    $$
    a_{m,1} x_1 + a_{m,2} x_2 + \cdots + a_{m,n} x_n = b_m
    $$

What a linear equation in $n$ variables represents is a given space in $\mathbb{R}^n$.

\begin{paragraph}{Question:}
What it the line passing through $(1,1)$ and $(-1,-3)$?
\end{paragraph}     
\begin{paragraph}{Answer:}
    We will use first principles to find the equation of the line.\\[1pt]
    Let the equation of the line be $y = mx + c$.
    
    We know that $(1,1)$ and $(-1,-3)$ lie on the line.\\[1pt]
    $$1 = m*1 + c \dots (1)$$
    $$-3 = m*(-1) + c \dots (2)$$

    Subtracting $(2)$ from $(1)$, we get:
    $$4 = 2m$$
    $$m = 2$$

    Substituting $m = 2$ in $(1)$, we get:
    $$1 = 2 + c$$
    $$c = -1$$
    
\end{paragraph}

\subsection*{Matrix}
Group of numbers(or equations, expressions, etc.) in rows and columns.
\begin{paragraph}{Example:}

    $$A = \begin{pmatrix}
        1 & 0 & -1 & 4 \\
        2 & 9 & 3 & 5 \\
        5 & 2 & 10 & 6 
    \end{pmatrix}$$
The above matrix has 3 rows and 4 columns.\\[1pt]
Entry in 2nd row, 3rd column $ = 3$

It can also be represented as $$A = (a_{ij})$$ 
where $a_{ij}$ refers to the element in A at the $i$th row and $j$th column

\end{paragraph}

\begin{paragraph}{Special Matrices:} Some special matrices:

    $\cdot$ \textbf{Zero Matrix:} 
    
    $$ A = O_{m \times n} = \begin{bmatrix}
        0 & . & . & . & 0 \\
        \cdot &&&& \cdot \\
        \cdot &&&& \cdot \\
        0 & . & . & . & 0
    \end{bmatrix}$$
    
    $\cdot$ \textbf{Square Matrix:} here $m = n$
        $$A_3 = \begin{bmatrix}
            a & b & c \\
            d & e & f \\
            g & h & i
        \end{bmatrix}$$

    $\cdot$ \textbf{Identity Matrix:} A square matrix with all diagonal element equal to 1 and non-diagonal elements equal to 0.
        $$ I_n = \begin{bmatrix}
            1 &&&&& 0 \\
            &1&&&&\\
            &&.&&&\\
            &&&.&&\\
            &&&&.&\\
            0&&&&&1
        \end{bmatrix}$$

    $\cdot$ \textbf{Diagonal Matrix:} A square matrix with all non-diagonal elements equal to 0.
        $$ D_n = \begin{bmatrix}
            d_1 &&&&& 0 \\
            &d_2&&&&\\
            &&.&&&\\
            &&&.&&\\
            &&&&.&\\
            0&&&&&d_n
        \end{bmatrix}$$
        where $d_i$ are generally non-zero, but not necessarily.
        
    $\cdot$ \textbf{Upper Triangular Matrix:} A square matrix with all non-diagonal elements below the diagonal equal to 0
        $$ B = \begin{bmatrix}
            b_1 &&&&& \star \\
            &b_2&&&&\\
            &&.&&&\\
            &&&.&&\\
            &&&&.&\\
            0&&&&&d_n
        \end{bmatrix}$$
        where $\star$ can be any number.

    $\cdot$ \textbf{Lower Triangular Matrix:} A square matrix with all non-diagonal elements above the diagonal equal to 0
        $$ C = \begin{bmatrix}
            c_1 &&&&& 0 \\
            &c_2&&&&\\
            &&.&&&\\
            &&&.&&\\
            &&&&.&\\
            \star&&&&&c_n
        \end{bmatrix}$$
        where $\star$ can be any number.
\end{paragraph}

\subsection*{Augmented Matrix}
Given a system of linear equations:
    $$
    3x_1 - 2x_2 + 4x_3 = 0
    $$
    $$
    2x_1 + x_2 + 3x_3 = 1
    $$
    $$
    5x_1 + x_2 - 2x_3 = -1
    $$
    
The augmented matrix is:
$$
\begin{bmatrix}
    3 & -2 & 4 & : & 0 \\
    2 & 1 & 3 & : & 1 \\
    5 & 1 & -2 & : & -1 
\end{bmatrix}
$$    

for a system of $m$ equations and $n$ variables:

    $$
    a_{1,1} x_1 + a_{1,2} x_2 + \cdots + a_{1,n} x_n = b_1
    $$
    $$
    a_{2,1} x_1 + a_{2,2} x_2 + \cdots + a_{2,n} x_n = b_2
    $$
    $$
    \vdots
    $$
    $$
    a_{m,1} x_1 + a_{m,2} x_2 + \cdots + a_{m,n} x_n = b_m
    $$

The augmented matrix would be:
$$
\begin{bmatrix}
    a_{1,1} & a_{1,2} & \cdots & a_{1,n} & : & b_1 \\
    a_{2,1} & a_{2,2} & \cdots & a_{2,n} & : & b_2 \\
    \vdots & \vdots & \ddots & \vdots & \vdots & \vdots \\
    a_{m,1} & a_{m,2} & \cdots & a_{m,n} & : & b_m
\end{bmatrix}
$$

\subsection*{Elementary Row Operations}
\begin{itemize}
    \item Interchange two rows: if $R_i$ and $R_j$ are two rows of a matrix $A$, then $R_i \leftrightarrow R_j$ 
    \item Multiply a row by a constant: if $R_i$ is a row of a matrix $A$ and $c$ is a constant, $R_i \rightarrow cR_i$
    \item Multiply a row by a constant and add it to another row: if $R_i$ and $R_j$ are two rows of a matrix $A$ and $c$ is a constant, $R_i \rightarrow R_i + cR_j$
\end{itemize}

Testing for a change

\end{document}
