\documentclass[12pt, oneside]{book}
\usepackage{amsmath, amssymb, amssymb, mathtools, graphicx, hyperref, titling}
\usepackage{tikz, caption, subcaption, enumitem, polyglossia}
\usepackage{tikz-3dplot, float, etoolbox, enumitem}
\usepackage[top=25mm, bottom=25mm, left=20mm, right=20mm]{geometry}
%\usepackage{showframe}

\makeatletter
\patchcmd{\@makechapterhead}{\vspace*{50\p@}}{}{}{}% Removes space above \chapter head
\patchcmd{\@makeschapterhead}{\vspace*{50\p@}}{}{}{}% Removes space above \chapter* head
\makeatother

\predate{}
\postdate{}
\date{}
\title{MT4214 - Algebraic Geometry}
\author{Nachiketa Kulkarni}
\pagenumbering{gobble}
\setmainfont{Comic Sans MS}

\begin{document}
\maketitle
\tableofcontents

\mainmatter
\chapter{Motivation - Cayley–Hamilton theorem}
\paragraph{Statement:} Every Square Matrix over a commutative ring satisfies its own Charectoristic Polynomial.

\paragraph{Proof:} Step 1: Let \(A\) be a diagonal matrix with \(\{\lambda_1, \lambda_2, \dots, \lambda_n\}\) as the diagonal elements.
Trivially, we can show that the Charecteristic Polynomial will be evaluated as follows:
\begin{align*}
	\chi_A (x) &= \det(A - xI_n)\\
	&= (\lambda_1 - x)(\lambda_2 - x)\dots(\lambda_n - x)\\
	&= 0
\end{align*}
Step 2: \(A\) is diagonalizable.
Then there exists matrices \(B, D\) such that \(A = BDB^{-1}\).
A property that will be used is as follows: \(\chi_A(x) = \chi_D(x)\).
Now, if we calculate the Charecteristic Polynomial for A:
\begin{align*}
	\chi_A (A) &= \det(A - xI_n)\\
	&= \chi_D(A)\\
	&= B \chi_D (D) B^{-1}\\
	&= 0
\end{align*}
Step 3: General A.
We know that diagonalizable matrices are dense in \(M_{n\times n}(\mathbb{C})\).

Consider the following function:
\[\phi : M_{n\times n}(\mathbb{C}) \rightarrow M_{n\times n}(\mathbb{C})\]
such that \(\phi(A) = \chi_A(A) = 0 \: \forall A \in \text{Diagonal Matrices}\).
The above function is a continuous function [Trust me bro].
Now, \(\{0\}\) is a closed set.
Therefore the pullback of a closed set will have to be a closed set as well.
But diagonal elements are dense in \(M_{n\times n}(\mathbb{C})\).
Therefore we use this to extend this to the entire topological space, \(M_{n\times n}(\mathbb{C})\).
\[\phi(A) = 0 \: \forall \: A\]
But this above arguement only for fields which are Cauchy Complete.
What about the charecteristic \(p\) fields.
There is no obvious topology, and hence no dense set.

\section{Zariski Topology on \(K^n\)}
Let \(K\) be an algebraicaly closed field. We want to define a topology on \(K^n\).

Define a ring \(A = K[X_1, X_2, \dots, X_n]\) is the ring of polynomial in \(n\) variables.
Now, choose an element \(f \in A\).
\[f:K^n \rightarrow K \text{ where } (a_1, a_2, \dots, a_n) \rightarrow f(a_1, a_2, \dots, a_n) \]
Now, we define a set function as follows:
\[Z(f) = \{(a_1, a_2, \dots, a_n) \: \vline \: f(a_1, a_2, \dots, a_n) = 0\}\]
Here, \(Z(f)\) can be empty.
Extending this to multiple functions:
\[Z(f_1, f_2, \dots, f_m) = \bigcap_{1 \leq i \leq m} Z(f_i)\]
Let \(I \subseteq A\) be an ideal.
\[f(I) = \bigcap_{p \in I} Z(p)\]

\paragraph{Noetherian Ring:} Let \(R\) be a commutative ring with unity.
\(R\) is Noetherian if every ideal of \(R\) is finitely generated.
\begin{center}
OR
\end{center}
\(R\) is Noetherian if every increasing sequence \(I_1 \subseteq I_2 \subseteq I_3 \subseteq \dots \) of ideals has a largest element.
\end{document}
