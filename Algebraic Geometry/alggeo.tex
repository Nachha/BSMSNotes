\documentclass[12pt, oneside]{book}
\usepackage{amsmath, amssymb, amssymb, mathtools, graphicx, hyperref, titling}
\usepackage{tikz, caption, subcaption, enumitem, polyglossia}
\usepackage{tikz-3dplot, float, etoolbox, enumitem}
\usepackage[mathrm=sym]{unicode-math}
\usepackage[top=25mm, bottom=25mm, left=20mm, right=20mm]{geometry}
%\usepackage{showframe}

\makeatletter
\patchcmd{\@makechapterhead}{\vspace*{50\p@}}{}{}{}% Removes space above \chapter head
\patchcmd{\@makeschapterhead}{\vspace*{50\p@}}{}{}{}% Removes space above \chapter* head
\DeclarePairedDelimiter{\spanset}{\langle}{\rangle}
\DeclareMathOperator{\Hom}{\mathrm{Hom}}
\DeclareMathOperator{\ob}{\mathrm{ob}}
\makeatother

\predate{}
\postdate{}
\date{}
\title{MT4214 - Algebraic Geometry}
\author{Nachiketa Kulkarni}
\pagenumbering{gobble}
\setmainfont{Comic Sans MS}

\begin{document}
\maketitle
\tableofcontents

\mainmatter
\chapter{Motivation - Cayley–Hamilton theorem}
\paragraph{Statement:} Every Square Matrix over a commutative ring satisfies its own Charectoristic Polynomial.

\paragraph{Proof:} Step 1: Let \(A\) be a diagonal matrix with \(\{\lambda_1, \lambda_2, \dots, \lambda_n\}\) as the diagonal elements.
Trivially, we can show that the Charecteristic Polynomial will be evaluated as follows:
\begin{align*}
	\chi_A (x) &= \det(A - xI_n)\\
	&= (\lambda_1 - x)(\lambda_2 - x)\dots(\lambda_n - x)\\
	&= 0
\end{align*}
Step 2: \(A\) is diagonalizable.
Then there exists matrices \(B, D\) such that \(A = BDB^{-1}\).
A property that will be used is as follows: \(\chi_A(x) = \chi_D(x)\).
Now, if we calculate the Charecteristic Polynomial for A:
\begin{align*}
	\chi_A (A) &= \det(A - xI_n)\\
	&= \chi_D(A)\\
	&= B \chi_D (D) B^{-1}\\
	&= 0
\end{align*}
Step 3: General A.
We know that diagonalizable matrices are dense in \(M_{n\times n}(\mathbb{C})\).

Consider the following function:
\[\phi : M_{n\times n}(\mathbb{C}) \rightarrow M_{n\times n}(\mathbb{C})\]
such that \(\phi(A) = \chi_A(A) = 0 \: \forall A \in \text{Diagonal Matrices}\).
The above function is a continuous function [Trust me bro].
Now, \(\{0\}\) is a closed set.
Therefore the pullback of a closed set will have to be a closed set as well.
But diagonal elements are dense in \(M_{n\times n}(\mathbb{C})\).
Therefore we use this to extend this to the entire topological space, \(M_{n\times n}(\mathbb{C})\).
\[\phi(A) = 0 \: \forall \: A\]
But this above arguement only for fields which are Cauchy Complete.
What about the charecteristic \(p\) fields.
There is no obvious topology, and hence no dense set.

\section{Zariski Topology on \(K^n\)}
Let \(K\) be an algebraicaly closed field. We want to define a topology on \(K^n\).

Define a ring \(A = K[X_1, X_2, \dots, X_n]\) is the ring of polynomial in \(n\) variables.
Now, choose an element \(f \in A\).
\[f:K^n \rightarrow K \text{ where } (a_1, a_2, \dots, a_n) \rightarrow f(a_1, a_2, \dots, a_n) \]
Now, we define a set function as follows:
\[Z(f) = \{(a_1, a_2, \dots, a_n) \: \vline \: f(a_1, a_2, \dots, a_n) = 0\}\]
Here, \(Z(f)\) can be empty.
Extending this to multiple functions:
\[Z(f_1, f_2, \dots, f_m) = \bigcap_{1 \leq i \leq m} Z(f_i)\]
Let \(I \subseteq A\) be an ideal.
\[f(I) = \bigcap_{p \in I} Z(p)\]

\paragraph{Lemma:} Let \(I \subseteq A\) be an ideal of A generated as follows: \(I = \spanset{f_1, f_2, \dots, f_n}\).
Then:
\begin{align*}
	Z(I) &= Z(\spanset{f_1, f_2, \dots, f_n})\\
	&= Z(\{f_1, f_2, \dots, f_n\})
\end{align*}
\paragraph{Proof:} We will prove this by showing that one set contains the other and vise=versa.\\[12pt]
Part 1: \(Z(\{f_1, f_2, \dots, f_n\}) \subseteq Z(I)\) as each of the \(f_i\) is always contained in the \(I\).\\[12pt]
Part 2: Consider an element \(f \in I\).
As the ideal \(I\) is generated by \({f_i}\), \(f\) can be written as a linear combination of \(f_i\)s, with coefficients in \(A\):
\[f = \sum_{i=1}^n c_i f_i\]
where, all \(c_i \in A\).
Now, if \(\bar{a} \in Z(\{f_1, f_2, \dots, f_n\})\), \(f_i(\bar{a}) = 0\) for all \(i\).
Therefore:
\begin{align*}
	f(\bar{a}) &= \sum_{i=1}^n c_i f_i(\bar{a})\\
	&= 0
\end{align*}
Hence, \(Z(\{f_1, f_2, \dots, f_n\}) \subseteq Z(I)\).
Therefore, \(Z(\{f_1, f_2, \dots, f_n\}) = Z(I)\).

\subsection{\(Z(I)\) form a Topology on \(K^n\)}
We define a topology on \(K^n\) by claiming that the closed sets in \(K^n\) are defined by the sets \(Z(I)\).
It satisifies the axioms of Topology as follows:
\begin{enumerate}
	\item \(\varnothing\) is in the topology: \(A\) is an ideal of itself.
	\(Z(A) = \varnothing\).
	\item \(K^n\) is in the topology: \(0\) polynomial also forms an ideal of \(A\).
	\(Z(0) = K^n\)
	\item Finite Union of \(Z(I)\) belong to the topology: Let \(I,J,IJ\) be Ideals of \(A\).\\[12pt]
	\textbf{Proposition:} \(Z(I) \cup Z(J) = Z(IJ)\).\\[12pt]
	\textbf{Proof:} Consider an \(a \in Z(I)\).
	This implies that \(f(a) = 0 \; \forall \; f \in I\).
	Using that we can say that \(a\) is a solution for any polynomial of the form \(f \cdot g\) where, \(f \in I\) and \(g \in A\).
	Now, similarly, consider a \(b\) in \(Z(J)\).
	It would be a solution for any polynomial of the form \(f \cdot g\) where, \(f \in J\) and \(g \in A\).
	Hence, \(a \in Z(I) \cup Z(J) \Rightarrow a \in Z(IJ)\)

	Consider an \(a \in Z(IJ)\).
	This implies that for all \(f \in I\) and \(g \in J\), the product \(f(a)g(a) = 0\).
	As \(A\) is an integral domain, one of the factors must be \(0\).
	Let \(a \notin I\).
	Then, there exists an \(f \in I\) such that \(f(a) \neq 0\).
	But, as \(a \in Z(IJ)\), for every \(g \in J\), \(f(a)g(a) = 0\).
	This implies \(g(a) = 0\) for every \(g \in J\).
	Hence, \(a \in Z(IJ) \Rightarrow Z(I) \cup Z(J) \)
	
	\item Arbitrary Intersection of \(Z(I)\) belong to the topology: let \(I, J, I+J\) be Ideals of \(A\).\\[12pt]
	\textbf{Proposition:} \(Z(I) \cap Z(J) = Z(I + J)\).\\[12pt]
	\textbf{Proof:} Consider an \(a \in Z(I) \cap Z(J)\).
	This implies that for any two polynomials \(f \in I \text{ and } g \in J\), the sum \((f + g)(a) = 0\).
	Therefore, \(a \in Z(I + J)\).

	Now, let \(a \in Z(I +J)\).
	Trivially, \(a \in Z(I) \text{ and } a \in Z(J)\).
\end{enumerate}
Therefore, \(Z(I)\) forms a topology on \(K^n\).
\(\mathbb{A}_k^n\) is defined as the vector space \(K^n\) with the Zariski Topology.
\paragraph{Exercise:} Let \(X\) be a Compact Hausdorff Topological Space.
\[C(X) = \{f: X \rightarrow \mathbb{R} \; \vline \; f \text{ is continuous.}\}\]
Show that \(C(X)\) is an \(\mathbb{R}\)-Algebra on \(X\) and that it forms a Commutative Ring.

\subsection{Category Theory}
\paragraph{Definition:} A Category \(C\) is a collection of objects \(\ob(C)\) and the following conditions:
\begin{enumerate}
	\item For any two elements \(x_1, y_1 \in \ob(C)\), there is a set of morphisms \(\Hom_C(x_1, y_1)\).
	(We can think of \(\ob(C)\) as dots and \(\Hom_C(x_1, y_1)\) an arrow (could be bidirectional) from \(x_1\) to \(y_1\)).
	\item Composition of Morphisms is defined as follows: For any \(x, y, z \in \ob(C)\)
	\begin{align*}
		\Hom(x,y) \times \Hom(y,z) &\rightarrow \Hom(x,z)\\
		f \times g &\mapsto g \circ f
	\end{align*}
\end{enumerate}
Which follow these axioms:
\begin{enumerate}
	\item If \(x, x', y, y' \in \ob(C)\), and \(x \neq x'\), \(y \neq y'\), then \(f \in \Hom(x,y)\) does not belong to \(\Hom(x', y')\).
	\item For any \(x,y,z \in ob(C)\), \(x \circ (y \circ z) = (x \circ y) \circ z\).
	\item For all \(x \in \ob(C)\), there exists identity map \(I_x \in \Hom(x,x)\) such that:
\end{enumerate}
with the following properties:
\begin{enumerate}
	\item If \(x \neq x'\) or \(y \neq y'\), then:
	\[\Hom_C (x,y) \cap \Hom_C (x',y') = \varnothing \]
	\item Composition is associative:
	\[(f \circ (g \circ h)) = ((f \circ g) \circ h) = (f \circ g \circ h)\]
	\item Identity maps: for every \(x,y \in \ob(C)\), there exist \(I_x(x,x) \in \Hom_C(x,x)\) and \(I_y(y,y) \in \Hom_C(y,y)\):
	\[f \circ I_x = f \text{ and } I_y \circ f = f\]
	And \(I_x \circ f\) doesn't make any sense as the set maps don't align.
\end{enumerate}
\paragraph{Example:} Groups, Rings, \(k\)-Algebra, \(C_{\mathrm{opp}}\).\\
\(C_{\mathrm{opp}}\) is constructed as:
\begin{itemize}
	\item Consider a Category \(C\).
	\item \(C_{\mathrm{opp}}\) contains the same points a \(C\).
	\item \(\Hom_{C_{\mathrm{opp}}(y,x)} = \Hom_C(x,y)\)
\end{itemize}

\subsection{Commutative Algebra}

\paragraph{Noetherian Ring:} Let \(R\) be a commutative ring with unity.
\(R\) is Noetherian if every ideal of \(R\) is finitely generated.
\begin{center}
OR
\end{center}
\(R\) is Noetherian if every increasing sequence \(I_1 \subseteq I_2 \subseteq I_3 \subseteq \dots \) of ideals has a largest element.

\paragraph{Theroem:} Show that the two definitions of a Noetherian Ring are equivalent.
I.e., If \(R\) has Ascending Chain Condition \(\Leftrightarrow\) All ideals of \(R\) are finitely generated.

\begin{enumerate}[label=Part \arabic*:, leftmargin=25pt, itemindent=\parindent, listparindent=\parindent]
	\item Consider a Commutative Ring \(R\) with a \(1\) and the Ascending Chain Condition.
	Let \(I\) be an Ideal which is not finitely generated.
	We can form the following chain:
	\begin{align*}
		I_0 &= 0 \\
		I_1 &= \spanset{f_1} \qquad& f_1 \in I \smallsetminus I_0 \\ 
		I_2 &= \spanset{f_1, f_2} \qquad& f_2 \in I \smallsetminus I_1 \\		
		\vdots\\
		I_n &= \spanset{f_1, f_2, \dots, f_n} \qquad& f_n \in I \smallsetminus I_n 
	\end{align*}
	\[I_0 \subsetneq I_1 \subsetneq I_2 \subsetneq \dots \subsetneq I_n \subsetneq \dots\]
	Now, as \(R\) has Ascending Chain Condition, \(I_n\) stabilizes, i.e., there will be no \(f\) to pick from \(I \smallsetminus I_m\) for some finite \(m\).
	Therefore, \(I\) is finitely generated.
	\item Consider a Commutative Ring \(R\) with a \(1\) and every ideal in \(R\) is finitely generated.
	Let there exist an Infinite Ascending Chain of Ideals, which does not stabilize:
	\[I_0 \subsetneq I_1 \subsetneq I_2 \subsetneq \dots \subsetneq I_n \subsetneq \dots\]
	As every ideal is finitely generated, consider the ideal:
	\[I = \bigcup_{n=0}^{\infty} I_n\]
	Let \(I\) be generated by \(\{f_1, f_2, \dots, f_m\}\).
	As each \(f_i \in I\), there exists a \(N_i\) such that \(f_i \in I_{N_i}\).
	Let \(N = \max(N_1, N_2, \dots, N_m)\).
	Then, for all \(i\), \(f_i \in I_N\).
	Therefore, \(I \subseteq I_N\).
	But, by construction, \(I_N \subseteq I\).
	Hence, \(I = I_N\).
	But, as the chain is strictly increasing, there exists a \(I_{N+1}\) such that \(I_N \subsetneq I_{N+1} \subseteq I\).
	This is a contradiction.
	Therefore, the chain stabilizes.
\end{enumerate}
\paragraph{Hilbert Basis Theorem:} Let \(R\) be a Noetherian Ring.
Then, the Polynomial Ring \(R[X]\) is also Noetherian.

\end{document}
use small k for fields instead of Capital K
