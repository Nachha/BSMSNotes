\documentclass[12pt, oneside]{book}
\usepackage{amsmath, amssymb, amssymb, mathtools, graphicx, hyperref, titling}
\usepackage{tikz, caption, subcaption, enumitem, polyglossia}
\usepackage{tikz-3dplot, float, etoolbox, enumitem}
\usepackage[top=25mm, bottom=25mm, left=20mm, right=20mm]{geometry}
%\usepackage{showframe}

\makeatletter
\patchcmd{\@makechapterhead}{\vspace*{50\p@}}{}{}{}% Removes space above \chapter head
\patchcmd{\@makeschapterhead}{\vspace*{50\p@}}{}{}{}% Removes space above \chapter* head
\makeatother

\predate{}
\postdate{}
\date{}
\title{Classical Field Theory}
\author{Nachiketa Kulkarni}
\pagenumbering{gobble}
\setmainfont{Comic Sans MS}

\begin{document}
\maketitle

\tableofcontents

\mainmatter
\pagenumbering{arabic}
\chapter{Principle of Relativity}
\paragraph{Intertial Frame of Reference:} A Frame of Reference such that a freely moving body, proceeds with constant velocity.
\section{Velocity of Propagation of Interaction}
The Velocity of Propagation of Interaction is the distance between two bodies divided by the interval by which the initial change in the first body reaches the second body.
It is a finite value and, if it has a maxima, then the motion of the particles of the bodies cannot exceed this maximum.

From Special Relativity, the maxima of the Velocity of Propagation is constant in all inertial frames and is the velocity of light in empty space.
Principle of relativity and the finiteness of Velocity of Propagation is called Principle of Relativity of Einstien.

In relativistic mechanics, time is not absolute.
Meanwhile, the combination of velocities is the same as in classical mechanics, and is the vector sum of the velocities of the individual velocities.
This is universal, and is applicable to the velocity of propagation of interaction.

By Michelson Morley experiment, the velocity of propagation of light is the same regardless of direction of propagation.

\section{Intervals}
An event is described by the "place" and "time" at which it occurred.
The place is described be its 3-dimensional coordinates for a particle.

Consider two frames \(K\) and \(K'\).
Now consider that an event has occurred at \((x_1,y_1,z_1)\) and at time, \(t_1\) in the \(K\) frame.
The information of this event travels at the speed of light and reaches and observer at \(x_2,y_2,z_2\) at time \(t_2\), in the same frame.
Then the distance between the event and the observer will be the distance covered by light in the time interval \((t_2 - t_1)\). Therefore:
\[\left(x_2 - x_1\right)^2 + \left(y_2 - y_1\right)^2 + \left(z_2 - z_1\right)^2 - c^2\left(t_2 - t_1\right)^2 = 0 \]
Now, if we consider the same event and observer in the frame \(K'\), the same relation will still hold true:
\[\left(x_2' - x_1'\right)^2 + \left(y_2' - y_1'\right)^2 + \left(z_2' - z_1'\right)^2 - c^2\left(t_2' - t_1'\right)^2 = 0 \]
Hence, if the position of any two events are \(ct_1, x_1, y_1, z_1\) and \(ct_2, x_2, y_2, z_2\), the "interval" between them is defined as:
\[s_{12} = \left[ c^2(t_2 - t_1)^2 - (x_2 - x_1)^2 - (y_2 - y_1)^2 - (z_2 - z_1)^2 \right]^{1/2}\]
Now, for two events infinitely close to each other:
\[ds^2 = c^2 dt^2 - x^2 - y^2 - z^2\]
Now, if in some other frame, \(K'\), the interval is \(ds'\), then the infinitesimals will be proportional:
\[ds^2 = \alpha ds'^2\]
where, \(\alpha\) can only depend on the relative velocity between the two frames.

Now, let us consider three different frames of reference moving with corresponding relative velocities to each other, \(K, K_1, K_2\) and \(V_1, V_2\) being the relative velocities of \(K_1, K_2\) wrt \(K\).
\begin{align*}
ds^2 &= a(V_1)ds_1^2 &
ds^2 &= a(V_2)ds_2^2 
\end{align*} 
Combining the two relations:
\[\frac{a(V_2)}{a(V_1)} = a(V_{12})\]
\(V_{12}\) does not only depend on the absolute value of \(V_1, V_2\), but also their directions.
But there is no angle dependence of direction on the value of \(a\) as shown above.
Therefore, \(a\) reduces to \(1\).

This shows that the interval between two events is the same in all inertial frames of reference.
This invarience shows that the velocity of light is constant (it is circular arguement, but ok). 

\subsection{Space-like and Time-like Intervals}
Now, consider two events in frame \(K\): \(t_1,x_1,y_1,z_1\) and \(t_2,x_2,y_2,z_2\).
We introduce the following notations:
\begin{align*}
t_{21} &= t_1 - t_2 &
l_{21}^2 &= (x_1 - x_2)^2 + (y_1 - y_2)^2 + (z_1 - z_2)^2
\end{align*}
Using this, the interval between the two events in \(K\) frame will be:
\[s_{12}^2 = c^2t_{12}^2 - l_{12}^2\]
Now, we will try to find a frame \(K'\) where the two events occur at the same point in space:
\[s_{12}'^2 = c^2t_{12}'^2 - l_{12}'^2\]
By invarience of the interval, we get the following relation:
\[c^2t_{12}^2 - l_{12}^2 = c^2t_{12}'^2 - l_{12}'^2\]
Here, \(l_{12}'^2 = 0\). Therefore:
\[s_{12}^2 = c^2t_{12}^2 - l_{12}^2 = c^2t_{12}'^2 > 0 \]
Therefore, the frame \(K'\) frame exists if the interval betweent the two events is a real number. Real intervals are said to be time-like.
If the interval between two events is time-like, then there exists a frame of reference such that the two events occur at the same point in space.
The time elapsed between the two events in the \(K'\) frame is as follows:
\[t_{12}' = \frac{1}{c}\sqrt{c^2t_{12}^2 - l_{12}^2} = \frac{s_12}{c}\]
If the event occurs in the same body, the interval is always time-like. 

Now, let us try to find when the event occurs at the same time.
Again, using the same method as before, we get:
\[s_{12}^2 = -l_{12}'^2 < 0\]
The above condition is only possible when the interval between the two events is imaginary.
\section{Proper Time}
 
\end{document}
