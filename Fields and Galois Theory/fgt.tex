\documentclass[12pt, oneside]{book}
\usepackage{amsmath, amssymb, amssymb, mathtools, graphicx, hyperref, titling}
\usepackage{tikz, caption, subcaption, enumitem, polyglossia}
\usepackage{tikz-3dplot, float, etoolbox, enumitem}
\usepackage[top=25mm, bottom=25mm, left=20mm, right=20mm]{geometry}
%\usepackage{showframe}

\makeatletter
\patchcmd{\@makechapterhead}{\vspace*{50\p@}}{}{}{}% Removes space above \chapter head
\patchcmd{\@makeschapterhead}{\vspace*{50\p@}}{}{}{}% Removes space above \chapter* head
\makeatother

\predate{}
\postdate{}
\date{}
\title{MT3164 - Fields and Galois Theory}
\author{Nachiketa Kulkarni}
\pagenumbering{gobble}
\setmainfont{Comic Sans MS}

\begin{document}
\maketitle
\tableofcontents

\mainmatter
\chapter{Basic Ring and Field Theory}
\section{Rings and Fields}
A Ring, \(R\), is a non-empty set with two operations \(+\) and \(.\) with the following properties:
\begin{itemize}
    \item \((R,+)\) is an abelian group.
    \item \(.\) is a binary operation on \(R\)
    \item \(.\) is associative.
    \item \(.\) distributes over \(+\).
\end{itemize}
\paragraph{Multiplicative Identity:} Some Rings have a Multiplicative Identity, \(1 \neq 0\) such that, \(\forall x \in R\):
\[x.1 = x = 1.x\]
\paragraph{Commutative Ring:} Where the multiplication operation is Commutative.\\
A field is a commutative ring with identity, where every non-zero element has an inverse.
\end{document}