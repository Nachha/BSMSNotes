\documentclass[12pt, oneside]{book}
\usepackage{amsmath, amssymb, amssymb, mathtools, graphicx, hyperref, titling}
\usepackage{tikz, caption, subcaption, enumitem, polyglossia}
\usepackage{tikz-3dplot, float, etoolbox, enumitem}
\usepackage[top=25mm, bottom=25mm, left=20mm, right=20mm]{geometry}
%\usepackage{showframe}

\makeatletter
\patchcmd{\@makechapterhead}{\vspace*{50\p@}}{}{}{}% Removes space above \chapter head
\patchcmd{\@makeschapterhead}{\vspace*{50\p@}}{}{}{}% Removes space above \chapter* head
\makeatother

\predate{}
\postdate{}
\date{}
\title{MT3434 - Topics in Number Theory}
\author{Nachiketa Kulkarni}
\pagenumbering{gobble}
\setmainfont{Comic Sans MS}

\begin{document}
\maketitle
\tableofcontents

\mainmatter
\chapter{Basics}
\section{Arithmetic Functions}
An Arithmetical function is any function defined as \(f: \mathbb{N} \rightarrow \mathbb{C}\).
\paragraph{Additive Arithmetic Funtion:} An Arithmetic function is Additive if for all relatively primes \(m,n \in \mathbb{N}\):
\[f(m \cdot n) = f(m) + f(n)\]
If the above function holds for all \(m,n \in \mathbb{N}\) then \(f\) is completely additive.

\paragraph{Multiplicative Arithmetic Funtion:} An Arithmetic function is Multiplicative if for all relatively primes \(m,n \in \mathbb{N}\):
\[f(m \cdot n) = f(m) \cdot f(n)\]
If the above function holds for all \(m,n \in \mathbb{N}\) then \(f\) is completely multiplicative.
\subsection{Examples:}
\begin{enumerate}
	\item \(\omega(x) =\) No. of distinct Prime divisors of x.\\
	\(\Rightarrow\) Additive, but not completely.
	\item \(\Omega(x) =\) No. of Prime divisors of x, counted with multiplicity.\\
	\(\Rightarrow\) Completely Additive.

\end{enumerate}
\end{document}
