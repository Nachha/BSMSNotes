\documentclass[12pt, oneside]{book}
\usepackage{amsmath, amssymb, amssymb, mathtools, graphicx, hyperref, titling}
\usepackage{tikz, caption, subcaption, enumitem, polyglossia}
\usepackage{tikz-3dplot, float, etoolbox, enumitem}
\usepackage[top=25mm, bottom=25mm, left=20mm, right=20mm]{geometry}
%\usepackage{showframe}

\makeatletter
\patchcmd{\@makechapterhead}{\vspace*{50\p@}}{}{}{}% Removes space above \chapter head
\patchcmd{\@makeschapterhead}{\vspace*{50\p@}}{}{}{}% Removes space above \chapter* head
\makeatother

\predate{}
\postdate{}
\date{}
\title{Rings and Modules}
\author{Nachiketa Kulkarni}
\pagenumbering{gobble}
\setmainfont{Comic Sans MS}

\begin{document}
\maketitle
\tableofcontents

\mainmatter
\chapter{Introduction to Rings}
\section{Definition of a Ring}
A ring \(R\) is a set with two binary operations, \(+\) and \(\times\), satisfying the following conditions:
\begin{itemize}
    \item \(\left(R,+\right)\) is an abelian group.
    \item \(\times\) is associative.
    \item \(\times\) distributes over \(+\).
\end{itemize}
A Ring is said to be commutative if \(a \times b = b \times a\) for all \(a, b \in R\).\\
A Ring is said to have a multiplicative identity if there exists an element \(1 \in R\) such that \(1 \times a = a \times 1 = a\) for all \(a \in R\).\\

\subsection{Examples}
\begin{itemize}
    \item \textbf{Trivial Ring:} Take any abelian group \((G,+)\) and define multiplication as \(a \times b = 0\) for all \(a, b \in G\), where \(0\) is the identity of the group.
    \item \textbf{Integers:} The set of integers \(\mathbb{Z}\) with usual addition and multiplication forms a ring.
    Also, the quotient group \(\mathbb{Z}/n\mathbb{Z}\) is a ring for any integer \(n\).
    \item \textbf{Hamiltonian Quaternions:} The set of quaternions \(\mathbb{H} = {1, i, j, k}\), where \(i^2 = j^2 = k^2 = -1\).
    \item \textbf{Polynomial Rings:} 
\end{itemize}
\end{document}