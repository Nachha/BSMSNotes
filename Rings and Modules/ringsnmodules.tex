\documentclass[12pt, oneside]{book}
\usepackage{amsmath, amssymb, amssymb, mathtools, graphicx, hyperref, titling}
\usepackage{tikz, caption, subcaption, enumitem, polyglossia}
\usepackage{tikz-3dplot, float, etoolbox, enumitem}
\usepackage[top=25mm, bottom=25mm, left=20mm, right=20mm]{geometry}
%\usepackage{showframe}

\makeatletter
\patchcmd{\@makechapterhead}{\vspace*{50\p@}}{}{}{}% Removes space above \chapter head
\patchcmd{\@makeschapterhead}{\vspace*{50\p@}}{}{}{}% Removes space above \chapter* head
\makeatother

\predate{}
\postdate{}
\date{}
\title{Rings and Modules}
\author{Nachiketa Kulkarni}
\pagenumbering{gobble}
\setmainfont{Comic Sans MS}

\begin{document}
\maketitle
\tableofcontents

\mainmatter
\chapter{Introduction to Rings}
\section{Definition of a Ring}
A ring \(R\) is a set with two binary operations, \(+\) and \(\times\), satisfying the following conditions:
\begin{itemize}
    \item \(\left(R,+\right)\) is an abelian group.
    \item \(\times\) is associative.
    \item \(\times\) distributes over \(+\).
\end{itemize}
A Ring is said to be commutative if \(a \times b = b \times a\) for all \(a, b \in R\).\\
A Ring is said to have a multiplicative identity if there exists an element \(1 \in R\) such that \(1 \times a = a \times 1 = a\) for all \(a \in R\).\\

\paragraph{Subrings:} A subset \(S\) of a ring \(R\) is called a subring if:
\begin{itemize}
    \item \(S\) is closed under addition and multiplication.
    \item \(S\) contains the additive identity \(0\) of \(R\).
    \item For every \(a \in S\), \(-a \in S\).
\end{itemize}
\subsection{Examples}
\begin{itemize}
    \item \textbf{Trivial Ring:} Take any abelian group \((G,+)\) and define multiplication as \(a \times b = 0\) for all \(a, b \in G\), where \(0\) is the identity of the group.
    \item \textbf{Integers:} The set of integers \(\mathbb{Z}\) with usual addition and multiplication forms a ring.
    Also, the quotient group \(\mathbb{Z}/n\mathbb{Z}\) is a ring for any integer \(n\).
    \item \textbf{Hamiltonian Quaternions:} The set of quaternions \(\mathbb{H} = {1, i, j, k}\), where \(i^2 = j^2 = k^2 = -1\).
    \item \textbf{Polynomial Rings:} Fix a commutative ring \(R\).
    The set of polynomials with coefficients in \(R\), denoted \(R[x]\), forms a ring with addition and multiplication defined as usual.
\end{itemize}
\section{Properties of Rings}
\paragraph{Proposition:} If \(R\) is a ring, then the following hold:
\begin{enumerate}
    \item \(0a = a0 = 0\) for all \(a \in R\).
    \item \((-a)b = a(-b) = -(ab)\) for all \(a, b \in R\).
    \item If the ring has a multiplicative identity \(1\), then it is unique.
    \item \((-1)a = -a\) for all \(a \in R\).
\end{enumerate}
\paragraph{More Definitions:} Consider a ring \(R\):
\begin{itemize}
    \item A non-zero element \(a \in R\) is called a \textbf{zero divisor} if there exists a non-zero \(b \in R\) such that either \(ab = 0\) or \(ba = 0\).
    \item Assume \(R\) has a multiplicative identity \(1\).
    An element \(a \in R\) is called a \textbf{unit} if there exists an element \(b \in R\) such that \(ab = ba = 1\).
    The set of all units in \(R\) is denoted by \(R^\times\).
    \item A Ring \(R\) with identity is called an \textbf{integral domain} if it has no zero divisors and \(1 \neq 0\).
\end{itemize}
\paragraph{Proposition:} If \(R\) is an integral domain, then the following hold:
\begin{enumerate}
    \item \(R^\times\) is a group under multiplication.
    \item \(R\) is a field if multiplication is commutative and every non-zero element is a unit, i.e., \(R^\times = R - \{0\}\).
    \item A zero divisor cannot be a unit and vice versa.
    \paragraph{Proof:} If \(a\) is a zero divisor, then there exists a non-zero \(b\) such that \(ab = 0\).
    Now, assume \(a\) is a unit, then there exists \(c\) such that \(ac = 1\).
    But:
    \[b = (ca)b = c(ab) = c0 = 0\]
\end{enumerate}
\section{Homomorphisms and Isomorphisms}
Let \(R\) and \(S\) be rings.
A \textbf{ring homomorphism} is a function \(\phi: R \to S\) such that:
\begin{enumerate}
    \item The map \(\phi\) preserves addition: \(\phi(a + b) = \phi(a) + \phi(b)\) for all \(a, b \in R\).
    \item The map \(\phi\) preserves multiplication: \(\phi(ab) = \phi(a) + \phi(b)\) for all \(a, b \in R\).
\end{enumerate}
The kernel of a ring homomorphism \(\phi\), \(\ker \phi\), is the set of elements in \(R\) that map to \(0\) in \(S\).
A bijective ring homomorphism is called a \textbf{ring isomorphism}, denoted by \(R \cong S\).
\end{document}