\documentclass[12pt, oneside]{book}
\usepackage{amsmath, amssymb, amssymb, mathtools, graphicx, hyperref, titling}
\usepackage{tikz, caption, subcaption, enumitem, polyglossia}
\usepackage{tikz-3dplot, float, etoolbox, enumitem}
\usepackage[top=20mm, bottom=20mm, left=15mm, right=15mm]{geometry}
%\usepackage{showframe}

\makeatletter
\patchcmd{\@makechapterhead}{\vspace*{50\p@}}{}{}{}% Removes space above \chapter head
\patchcmd{\@makeschapterhead}{\vspace*{50\p@}}{}{}{}% Removes space above \chapter* head
\makeatother

\predate{}
\postdate{}
\date{}
\title{MT2123 - Advanced Linear Algebra}
\author{Nachiketa Kulkarni}
\pagenumbering{gobble}
\setmainfont{Comic Neue}

\begin{document}
\maketitle
\tableofcontents

\mainmatter
\chapter{Fields and Vector Spaces}
\section{Groups}
\paragraph{Definition} A group \( \langle G, \ast \rangle \) is a set \( G \) with a binary operation \( \ast \) such that the following axioms are satisfied:
\begin{enumerate}
    \item Closure: For all \( a, b \in G \), \( a \ast b \in G \).
    \item Associativity: For all \( a, b, c \in G \), \( a \ast (b \ast c) = (a \ast b) \ast c \).
    \item Identity Element: There exists an element \( I \in G \) such that for all \( I \in G \), \( a \ast I = I \ast a = a \). Here, \( I \) is called as the identity element of \( \ast \) in \( G \).
    \item Inverse: corresponding to every element \( a \in G \), there exists an element \( a' \in G \) such that \( a \ast a' = a' \ast a = I \). Here, \( a' \) is called as the inverse of \( a \) in \( G \).
\end{enumerate}
\section{Rings}
\paragraph{Definition} A ring \( \langle R, +, \cdot \rangle \) is a set \( R \) with two binary operations \( + \) and \( \cdot \), which we call addition and multiplication, such that the following axioms are satisfied:
\begin{enumerate}
    \item \( \langle R, + \rangle \) is an abelian/commutative group. 
    \item Multiplication is associative: For all \( a, b, c \in R \), \( a \cdot (b \cdot c) = (a \cdot b) \cdot c \).
    \item Distributive Property: For all \( a, b, c \in R \), the Left Distributive Law, \( a \cdot (b + c) = a \cdot b + a \cdot c \) and Right Distributive Law, \( (a + b) \cdot c = a \cdot c + b \cdot c \).
\end{enumerate}
\section{Fields}
\paragraph{Definition} A field \( \langle F, +, \cdot \rangle \) is a set \( F \) with two binary operations \( + \) and \( \cdot \), which we call addition and multiplication, such that the following axioms are satisfied:
\begin{enumerate}
    \item Closure: For all \( a, b \in F \), \( a + b \in F \) and \( a \cdot b \in F \).
    \item Associativity: For all \( a, b, c \in F \), \( a + (b + c) = (a + b) + c \) and \( a \cdot (b \cdot c) = (a \cdot b) \cdot c \).
    \item Commutativity: For all \( a, b \in F \), \( a + b = b + a \) and \( a \cdot b = b \cdot a \).
    \item Identity Elements: There exist two elements \(I, O \in F\) such that for all \( a \in F \), \(I \cdot a = a \) and \( O + a = a \). Here, \( I \) is called as the multiplicative identity and \( O \) is called as the additive identity.
    \item Additive Inverse: For all \( a \in F \), there exists an element \( -a \in F \) such that \( a + (-a) = O \). Here, \( -a \) is called as the additive inverse of \( a \).
    \item Multiplicative Inverse: For all \( a \neq O \in F \), there exists an element \( a^{-1} \in F \) such that \( a \cdot a^{-1} = I \). Here, \( a^{-1} \) is called as the multiplicative inverse of \( a \).
    \item Distributivity: For all \( a, b, c \in F \), the Left Distributive Law, \( a \cdot (b + c) = a \cdot b + a \cdot c \) and Right Distributive Law, \( (a + b) \cdot c = a \cdot c + b \cdot c \).
\end{enumerate}
\subsection{Detour 1 - Finite Fields}
\paragraph{Theorem 1:} \( \mathbb{Z}_n \) is a field if and only if \( n \) is a prime number.
\paragraph{Proof:} Let \( n \) be any Positive Integer.
We can trivially show that for all \( n \), Closure and Associativity of addition and multiplication, Additive Identity, Multiplicative Identity and Additive Inverse, Distributivity rules are satisfied.
The only property we need to show is the existence of Multiplicative Inverse.
\begin{enumerate}[leftmargin=2.5cm, label=Case \arabic*:]
    \item \( n \) is a composite number with factors \( a, b \). 
Assume \( \mathbb{Z}_n \) is a field.
As, \( a \text{ and } b \) are factors of \( n \), they are less the \( n \), and hence belong to \( \mathbb{Z}_n \).
Now, \( a \cdot b = 0 \) in \( \mathbb{Z}_n \), but the product of two non-zero elements cannot be zero in a field. Hence, \( \mathbb{Z}_n \) cannot be a field, when \(n\) is composite.
    \item \( n \) is a prime number. We need to show that for all \( a \neq 0 \in \mathbb{Z}_n \), there exists an element \( a^{-1} \in \mathbb{Z}_n \) such that \( a \cdot a^{-1} = 1\).\\
Let us take an element \( a \in \mathbb{Z}_n \) such that \( a \neq 0 \).
Now, we know that the GCD of \( a \) and \( n \) is 1.
Hence, by Bézout's Identity, there exist integers \( x, y \) such that \( ax + ny = \gcd(a,n) = 1 \).
Applying modulo \( n \), we get \( ax \equiv 1 \pmod{n} \).
Hence, \( x \) is the multiplicative inverse of \( a \) in \( \mathbb{Z}_n \).
\end{enumerate}
\section{Vector Spaces}
\paragraph{Definition} A Vector Space, defined over a field \( F \) of scalers, is a set of objects \(V\) with two operations, Vector addition and Scalar Multiplication and follows the following axioms:
\begin{enumerate}
    \item Closure: For all \( u,v \in V \), \(u + v \in V \).
    \item Commutative: for all \( u,v \in V \), \(u + v = v + u \).
    \item Associativity: For all \( u,v,w \in V \), \(u + (v + w) = (u + v) + w \).
    \item Additive Identity: There exists an element \( O \in V \) such that for all \( u \in V \), \(O + u = u \).
    \item Additive Inverse: For all \( u \in V \), there exists an element \( -u \in V \) such that \(u + (-u) = O \).
    \item Scalar Multiplication: For all \( a \in F \) and \( u \in V \), \(a \cdot u \in V \). It has the following properties:
    \begin{enumerate}
        \item Multiplication by \(I\): If \(I\) is the Multiplicative identity of \( F \), then \(I \cdot u = u \).
        \item Unambiguous: for all \( a,b \in F \) and \( u \in V \), \( (a \cdot b) \cdot u = a \cdot (b \cdot u) \).
        \item Distributive: For all \( a \in F \) and \( u,v \in V \), \( a \cdot (u + v) = a \cdot u + a \cdot v \) and \( (a + b) \cdot u = a \cdot u + b \cdot u \).
        \item Distributive: For all \( a,b \in F \) and \( u \in V \), \( (a + b) \cdot u = a \cdot u + b \cdot u \).
    \end{enumerate}
\end{enumerate}
\section{Subspaces}
long not wall of text
\chapter{Linear Transformations}
long wall of text incoming

\end{document}