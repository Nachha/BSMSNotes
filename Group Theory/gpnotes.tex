\documentclass[12pt, oneside]{book}
\usepackage{amsmath, amssymb, amssymb, mathtools, graphicx, hyperref, titling}
\usepackage{tikz, caption, subcaption, enumitem, polyglossia}
\usepackage{tikz-3dplot, float, etoolbox, enumitem}
\usepackage[top=25mm, bottom=25mm, left=20mm, right=20mm]{geometry}
%\usepackage{showframe}

\makeatletter
\patchcmd{\@makechapterhead}{\vspace*{50\p@}}{}{}{}% Removes space above \chapter head
\patchcmd{\@makeschapterhead}{\vspace*{50\p@}}{}{}{}% Removes space above \chapter* head
\makeatother

\DeclarePairedDelimiter{\evdel}{\langle}{\rangle}
\newcommand{\ev}{\evdel}

\newcommand{\im}{\text{Im }}
\newcommand{\given}{\: \vline \:}

\predate{}
\postdate{}
\date{}
\title{MT2213 - Group Theory}
\author{Nachiketa Kulkarni}
\pagenumbering{gobble}
%\setmainfont{Times New Roman}

\begin{document}
\maketitle
\tableofcontents

\mainmatter
\chapter{Definitions}
\section{Groups}
A non-empty set G is a group, is considered to be a group with an operation \(\star\) if to every pair \( \left(x,y\right) \in G \times G\) and element \(x \star y \in G\) is assigned, satisfying the following axioms:
\begin{enumerate}
	\item \textbf{Associativity:} \(\forall\text{ } x,y,z \in G\), \(x \star \left(y \star z\right) = \left(x \star y \right) \star z = x \star y \star z \)
	\item \textbf{Existence of Identity:} There exists an element \(e \in G\) such that \(e \star g = g \star e = g\)
	\item \textbf{Existence of Inverse:} For every element \(x \in G\) there exists an element \(x^{-1} \in G\) such that \(x \star x^{-1} = e = x^{-1} \star x \), where \(e \in G\) is the identity element of the group.
\end{enumerate}
It is represented as \( \left(G, \star \right) \).
Some properties of groups:
\begin{enumerate}
	\item \textbf{Uniqueness of Identity:} The identity element of a group is unique.
	      Consider \(e_1, e_2 \in G, e_1 \neq e_2\) and both are identity elements.
	      Let \(x \in G\), then \( e_1 \star x = e_2 \star x = x \).
	      This also implies that \(e_1 = e_2\), hence the identity element is unique.
	\item \textbf{Uniqueness of Inverse:} The inverse of an element in a group is unique.
	      Consider \(x \in G\), and \(y_1, y_2 \in G\) are inverses of \(x\).
	      Then, \(x \star y_1 = e = y_1 \star x\) and \(x \star y_2 = e = y_2 \star x\).
	      Now, \(y_1 = y_1 \star e = y_1 \star \left(x \star y_2\right) = \left(y_1 \star x\right) \star y_2 = e \star y_2 = y_2\).
	      Hence, the inverse of an element is unique.
\end{enumerate}

\subsection{Examples:}
\begin{enumerate}
	\item \( \left( \mathbb{Z}, +\right) \) is a group:
	      \begin{enumerate}
		      \item Associativity: Addition is associative.
		      \item Identity: \(0\) is the identity. Let \(x \in Z\). Now \(0 + x = x + 0 = x\). Hence, it is an identity.
		      \item Inverse: Let \(x \in \mathbb{Z}\). Now, \(x + (-x) = (-x) + x = 0\), where \(0\) is the additive identity.
	      \end{enumerate}
	\item \(\left( \mathbb{Q}^{+}, \times \right)\) is a group:
	      \begin{enumerate}
		      \item Associativity: Multiplication is associative.
		      \item Identity: \(1\) is the identity:
		            Let \(x \in \mathbb{Q}^{+} \). Now, \(1 \times x = x \times 1 = x\). Hence, it is an identity.
		      \item Inverse: Let \(x \in \mathbb{Q}^{+} \), Now, \(x \times \frac{1}{x} = \frac{1}{x} \times x = 1\), where 1 is the multiplicative identity.
	      \end{enumerate} \break
	\item \(\left(GL(n,\mathbb{R}), \times \right)\) is a group, where \(\times\) is matrix multiplication (or combination of linear transformations):
	      \begin{enumerate}
		      \item Associativity: Matrix multiplication is associative.
		      \item Identity: \(I_n\) is the identity matrix.
		      \item Inverse: Let \(A \in GL(n,\mathbb{R})\), then \(A \times A^{-1} = A^{-1} \times A = I_n\).
	      \end{enumerate}
\end{enumerate}
Check if:
\begin{enumerate}
	\item \( \left(\mathbb{R}, \times \right)\) is a group or not.

	      \(0 \in \mathbb{R}\), \(0\) does not have an inverse. Hence, it is not a group.

	\item \(\left(\mathbb{C}, \times\right)\) is a group or not.

	      \(0 \in \mathbb{C}\), \(0\) does not have an inverse. Hence, it is not a group.

	\item \(\left(\mathbb{R}/\{0\}, \times \right)\) is a group or not.

	      Yes its a group:
	      \begin{enumerate}
		      \item Associativity: Multiplication is associative.
		      \item Identity: 1 is an identity:
		            Let \(x \in \mathbb{R}/\{0\} \). Now, \(1 \times x = x \times 1 = x\). Hence, it is an identity.
		      \item Inverse: Let \(x \in \mathbb{R}/\{0\} \), Now, \(x \times \frac{1}{x} = \frac{1}{x} \times x = 1\), where 1 is the multiplicative inverse.
	      \end{enumerate}

	\item \(\left(\mathbb{C}/\{0\}, \times \right)\) is a group or not.

	      Yes it is a group:
	      \begin{enumerate}
		      \item Associativity: Multiplication is associative.
		      \item Identity: 1 is an identity:
		            Let \(x \in \mathbb{C}/\{0\} \). Now, \(1 \times x = x \times 1 = x\). Hence, it is an identity.
		      \item Inverse: Let \(x \in \mathbb{C}/\{0\} \), Now, \(x \times \frac{1}{x} = \frac{1}{x} \times x = 1\), where 1 is the multiplicative inverse.
	      \end{enumerate}
\end{enumerate}
\subsection{Abelian Groups}
A group \( \left(G, \star \right) \) is said to be abelian if the operation \(\star\) is commutative, i.e., \(x \star y = y \star x\), \(\forall x,y \in G\).

\subsection{Conjugate}
Consider a group \( \left(G, \star \right) \).
For \(x,y \in G\), \(y\) is said to be conjugate of \(x\) if there exists an element \(a \in G\) such that:
\[ y = a \star x \star a^{-1} \]
\paragraph{Note:} For a given \(a\), the conjugate of \(x\) is unique.
i.e., if we consider conjugate to be a function \(f_a\), then \(f_a\) is a bijection.

\subsection{Order of a Group}
The order of a group \(G\) is the number of elements in the group.
It is denoted by \(|G|\).
A group \(G\) is said to be finite if the number of elements in it is finite.
Otherwise, it is said to be infinite.

\subsection{Cyclic Group}
A group \( \left(G, \star \right) \) is said to be cyclic if there exists an element \(a \in G\) such that every element of \(G\) can be written as a power of \(a\).
Let, \(G = \ev{g}\) by a cyclic group of order \(n\).
Then, \(G = \{e, g, g^2, \ldots, g^{n-1}\}\).
\subsubsection{Properties of Cyclic Groups}
\begin{enumerate}
	\item All cyclic groups are abelian.
	\item \(n = \min{ \left\{m \in \mathbb{N} \: \vline \: g^m = 1\right\} }\).

	      \paragraph{Proof:} As the order of \(G\) is finite, there exists \(a,b \in \mathbb{N}\) such that \(g^a = g^b\).
	      This implies: \(g^{a-b} = 1\).
	      \[ \therefore \exists \: n := \min{ \left\{m \in \mathbb{N} \: \vline \: g^m = 1\right\} } \]

	\item
\end{enumerate}


\subsection{Sub-groups}
Consider a group \( \left(G, \star \right) \).
A non-empty subset \(H\) is a subgroup of \(G\) if \(H\) is a group with the same operation \(\star\) as \(G\).
It is represented as \(H \leq G\).

A few properties of subgroups:
\begin{enumerate}
	\item \textbf{Identity:} The identity element of \(G\) is also the identity element of \(H\).
	\item \textbf{Inverse:} If \(x \in H\), then \(x^{-1} \in H\).
\end{enumerate}

\subsubsection{Minimal and Maximal Subgroups}
The Minimal subgroup \(U \neq 1\) is known as a minimal subgroup of group \(G\) if no other non-trivial subgroup of \(G\) is contained in \(U\).

The Maximal subgroup \(U \neq G\) is known as the maximal subgroup of group \(G\) if \(U\) is not contained in any other subgroup of \(G\).

\subsubsection{Theorem}
Let \(A\) and \(B\) be subgroups of \(G\).
Then \(AB\) is a subgroup of \(G\) if and only if \(AB=BA\).

\paragraph{Proof:} From \( AB \leq G \) we get:
\[ \left(AB \right) = \left(AB\right)^{-1} = B^{-1} A^{-1} = BA \]
If \(BA = AB\):
\[\left(AB\right) \left(AB\right) = A\left(BA\right) B = A\left(AB\right)B = AABB = AB \]
and
\[ \left(AB\right)^{-1} = B^{-1} A^{-1} = BA = AB\]
Therefore, \(AB \leq G\)

\subsubsection{Theorem}
Let \(A\) and \(B\) be finite subgroups of \(G\).
Then, \[ \left| AB \right| = \frac{ \left| A \right| \left| B \right|}{\left| A \cap B \right|} \]

\paragraph{Proof:} If we consider an equivalence relation on the Cartesian Product \( A \times B \):
\[ \left( a_1, b_1 \right) \sim \left( a_2, b_2 \right) \Leftrightarrow a_1 b_1 = a_2 b_2 \]
Then \(\left|AB\right|\) is the number of equivalence classes in \(A \times B\).
Let \( \left(a_1, b_1\right) \in A \times B \).
The equivalence class:
\[ \{ \left(a_2, b_2\right) \: \vline \: a_1 b_1 = a_2 b_2 \} \]
which contains exactly \(\left|A \cap B\right|\) elements:
\begin{align*}
	a_2 b_2 = a_1 b_1 & \Leftrightarrow a_1^{-1} a_2 = b_1 b_2^{-1}                                           \\
	                  & \Leftrightarrow a_2 = a_1 d \text{ and } b_2 = d b_1 \text{ for some } d \in A \cap B
\end{align*}

\subsection{Cosets}
Let \(\left( G, \star \right) \) be a group.
Consider \(H\) be a subgroup of \(G\) and \(a \in G\).
The subset \(aH = \{ah \: \vline \: h \in H\}\) is known as the left coset of \(H\) containing \(a\).
Similarly, the subset \( Ha = \{ah \: \vline \: h \in H\} \) is known as the right coset of \(H\) containing \(a\).

\subsubsection{Properties of Cosets}
\begin{enumerate}
	\item The application \(Hx \rightarrow \left(Hx \right)^{-1} = x^{-1} H \) defines a bijective relation from the set of Right Cosets of \(H\) to the set of Left Cosets of \(H\).
	\item If the set of Right Cosets of \(H\) in \(G\) is finite, then the number of Right Cosets of \(H\) in \(G\) is called the inedex of \(H\) in \(G\).
	\item One of the cosets is the subgroup \(H\) itself. \(eH = He = H\), where \(e\) is the identity element of the group \(G\).
	\item For all \(x \in G \), as \( x = ex \in Hx \), the right cosets of \(H\) cover the set \(G\).
	\item For \(x,y \in G \),
	      \[ Hx = Hy \Leftrightarrow yx^{-1} \in H \Leftrightarrow y \in Hx \]
	      Hence, any two right cosets are either disjoint or equal.
\end{enumerate}

\subsubsection{Lagrange's Theorem}
Let \(H\) be a subgroup of the finite group \(G\).
Then,
\[\left|G\right| = \left|H\right|\left|G:H\right| \]
i.e., \(|H|\) and \(|G:H|\) are divisors of \(H\).\\
As a consequence, we get the following:
\begin{quote}
	For every finite \(G\) and every \(g \in G\), the order of \(g\) divides \(|G|\).
\end{quote}

\subsubsection{Transversal Set}
Let \(H\) be a subgroup of \(G\).
A set \(S\) is considered as the transversal set of \(H\) in \(G\) if \(S\) contains exactly one element from each right coset of \(H\) in \(G\).
Similarly, the left transversal set of \(H\) in \(G\) contains exactly one element from each left coset of \(H\) in \(G\).
\paragraph{Theorem:} Let \(S \subseteq G\).
Then, \(S\) is a transversal set of \(H\) in \(G\) if and only if \(G = SH\) and \(s t^{-1} \notin H\) for all \(s \neq t\) and \(s,t \in S\).

\subsubsection{Dedekind Identity}
Let \(G = AB\) where \(A,B \leq G\).
Then every subgroup \(H\) of \(G\), such that \(A \leq H \leq G\) has the following property:
\[ H = A \left( H \cap B \right) \]

\section{Homomorphisms and Normal Subgroups}
Let \(G\) and \(H\) be groups.
A map \(\phi: G \rightarrow H\) is said to be a homomorphism if:
\[ \phi \left( x \star y \right) = \phi \left( x \right) \star \phi \left( y \right) \]
for all \(x,y \in G\).
Let \( \phi: G \rightarrow H \).
Let, \(X \subseteq G\) and \(Y \subseteq H\).
Also, let \(e_G\) is the identity element of \(G\) and \(e_H\) is the identity element of \(H\).
Then, we define the following:
\begin{itemize}
	\item \(\phi \left(X\right) := \{ \phi \left(x\right) \: \vline \: x \in X \} \)
	\item \(\phi^{-1} \left(Y\right) := \{ x \in G \: \vline \: \phi \left(x\right) \in Y \} \)
	\item \(\ker{\phi} := \{ x \in G \: \vline \: \phi \left(x\right) = e_H \} \)
	\item \( \im \phi := \phi\left(G\right)\)
\end{itemize}
\subsection{Properties of Homomorphisms}
\begin{enumerate}
	\item If the homomorphism \(\phi\) is bijective, then the inverse map \(\phi^{-1}\) is also a homomorphism.
	      \paragraph{Proof:} Let \(x,y \in H\).
	      \begin{align*}
		      \phi^{-1} \left( x \right) \star \phi^{-1} \left( y \right)                   & = \phi^{-1} \left( x \star y \right)                                                                        \\
		      \phi \left( \phi^{-1}\left( x \right) \star \phi^{-1}\left( y \right) \right) & = \phi \left( \phi^{-1} \left(x\right)\right) \star \phi \left( \phi^{-1} \left(y\right)\right) = x \star y
	      \end{align*}
	\item \(\phi\left(e_G\right) = e_H\)
	\item \(\phi\left(x^{-1}\right) = \left(\phi \left(x\right)\right)^{-1}\)
	\item \(\phi\left(\ev{X}\right) = \ev{\phi\left(X\right)}\)
	\item Let \(N = \ker \phi\). Then for all \(x \in G\)
	      \[ Nx = \{y \in G \given \phi \left(x\right) = \phi \left(y\right) \} = xN \]
	      \paragraph{Proof:}
	      \begin{alignat*}{3}
		      \phi \left(x\right) = \phi \left(y\right) & \Longleftrightarrow \phi \left(y\right) \left(\phi \left(x\right)\right)^{-1} = 1 &  & \Longleftrightarrow \phi \left(y\right) \phi\left(x^{-1}\right) \\
		                                                & \Longleftrightarrow \phi \left( yx^{-1} \right) = 1                               &  & \Longleftrightarrow yx^{-1} \in N                               \\
		                                                & \Longleftrightarrow y \in Nx                                                      &  &                                                                 \\
		  \end{alignat*}
\end{enumerate}
\subsection{Normal Subgroups}
A subgroup \(N\) of a group \(G\) is said to be normal if for all \(x \in G\), \(Nx = xN\).
We write \(N \trianglelefteq G\).

If \(N \trianglelefteq G\), then the set of left cosets and right cosets of \(N\) in \(G\) are the same.
Another way to define normal subgroups is, \( \forall x \in G\):
\[ Nx = xN \Longleftrightarrow N = x^{-1} Nx \Longleftrightarrow N = N^x \]

\paragraph{Trivial Normal Subgroups:} \(G\) and \( \{e\} \), where \{e\} is the identity are the trivial normal subgroups.
\paragraph{Simple Normal:} For a group, \(G\) if its only normal subgroups are trivial, then it is said to be trivial.

\subsubsection{Properties of Normal Subgroups:}
\begin{itemize}
	\item For every homomorphism \(\phi\) of \(G\), the image of any normal subgroup of \(G\) is normal in \(\phi\left(G\right)\).
	\item The product and intersection of two normal subgroups of \(G\) are also normal in \(G\).
	\item Let \(H\) be a subgroup of \(G\) and \(N\) be a normal subgroup of \(G\).
	      Then \(H \cap N\) is normal in \(H\).
	\item If \(H\) is a subgroup of \(G\).
	      Then \[U_G = \bigcap\limits_{g \in G} U^g\]
	\item Let \(X \subseteq G\). Then, \(\ev{X^G}\) is the smallest normal subgroup of \(G\) containing \(X\).
\end{itemize}

\end{document}
