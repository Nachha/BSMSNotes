\documentclass[12pt, oneside]{book}
\usepackage{amsmath, amssymb, amssymb, mathtools, graphicx, hyperref, titling}
\usepackage{tikz, caption, subcaption, enumitem, polyglossia}
\usepackage{tikz-3dplot, float, etoolbox, enumitem}
\usepackage[top=25mm, bottom=25mm, left=20mm, right=20mm]{geometry}
%\usepackage{showframe}

\makeatletter
\patchcmd{\@makechapterhead}{\vspace*{50\p@}}{}{}{}% Removes space above \chapter head
\patchcmd{\@makeschapterhead}{\vspace*{50\p@}}{}{}{}% Removes space above \chapter* head
\makeatother

\predate{}
\postdate{}
\date{}
\title{MT3244 - Calculus on Manifolds}
\author{Nachiketa Kulkarni}
\pagenumbering{gobble}
\setmainfont{Comic Sans MS}

\begin{document}
\maketitle
\tableofcontents

\mainmatter
\chapter{Parameterized Curves and Surfaces (in \(\mathbb{R}^n\))}
\section{Paramterized Curves}
\paragraph{Definition} A parameterized curve is a continiuous function
\[f: I \rightarrow  \mathbb{R}^n\]
where \(I\) is an open set in \(\mathbb{R}\).

\subsection{Examples}
\begin{enumerate}
	\item Straight line: \(f: \mathbb{R} \rightarrow \mathbb{R}^2 \) defined as \(f(\alpha) = (\alpha, m\alpha + c)\) where \(m\) and \(c\) are constants.
	\item Parabola: \(f: \mathbb{R} \rightarrow \mathbb{R}^2 \) defined as \(f(\alpha) = (\alpha, m\alpha^2)\) where \(m\) is a constant.
	\item Circle: \(f: \mathbb{R} \rightarrow \mathbb{R}^2 \) defined as \(f(\alpha) = (\cos\alpha, \sin\alpha)\).
	\begin{itemize}
		\item Function is bounded.
		\item There is no polynomial paramterization of this curve.
		\item Rational Parameterizations exists:
		\[f(t) = \left(\frac{t^2-1}{t+1}, \frac{2t}{t+1}\right)\]
	\end{itemize}
	\item Not involving Modulus: \(f: \mathbb{R} \rightarrow \mathbb{R}^2 \) defined as \(f(\alpha) = (\alpha^3, \alpha^2)\).
\end{enumerate}
\subsection{Differentiation Curves:}
A function \(f\) which is differentiable for a \(t \in I\) is called a Parameterized Differentiable Curves [for all possible paramterizations].
Let \(f(t) = (f_1(t), f_2(t), \dots, f_n(t))\), then its derivative is defined as: \(f'(t) = (f'_1(t), f'_2(t), \dots, f'_n(t))\).
This is used to define the direction of the tangent at a point \(f(t)\) as \(f'(t)\).
\paragraph{Regular Curve:} A differentiable curve such that the tangent vector is non-zero for all \(t\in I\)

\section{Paramterized Surfaces}
\paragraph{Definition} A parameterized curve is a continuous function
\[f: I \rightarrow \mathbb{R}^n\]
where \(I\) is an open set in \(\mathbb{R}^2\).
In our case, \(n = 3\).
\subsection{Examples}
\begin{enumerate}
	\item Affine Linear Transformation: \(f: \mathbb{R}^2\ \rightarrow \mathbb{R}^3\) defined as \(f(u,v) = (p_1 + q_1 u + r_1 v, p_2 + q_2 u + r_2 v, p_3 + q_3 u + r_3 v)\).
	Let \(q = (q_1, q_2, q_3)\) and \(r = (r_1, r_2, r_3)\) are:
	\begin{itemize}
		\item \(0\): The space is a fixed point.
		\item Linearly Dependent: The space is a line.
		\item Linearly Independent: The space is a plane.
	\end{itemize}
	\item Sphere: \(f: \mathbb{R}^2 \rightarrow \mathbb{R}^3\) defined as \(f(u,v) = (\cos u \cos v, \cos u \sin v, \sin u)\).
	There is no polynomial parameterization to a sphere.
	\item Cone: \(f: \mathbb{R}^2 \rightarrow \mathbb{R}^3\) defined as \(f(u,v) = (v \cos u, v \sin u, v)\).
	Polynomial Parameterization exists as follows:
	\[(u,v) \rightarrow (u^2 -v^2, 2v, u^2 + v^2)\]
\end{enumerate}
\subsection{Differentiable Surfaces}
\end{document}
